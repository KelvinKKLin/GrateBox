\documentclass{article}

\usepackage{tabularx}
\usepackage{booktabs}

\title{SE 3XA3: Problem Statement\\Genetic Cars}

\author{Team 8, Grate
		\\ Kelvin Lin (linkk4)
		\\ Eric Chaput (chaputem)
		\\ Jin Liu (liu456)
}

\date{}

%% Comments

\usepackage{color}

\newif\ifcomments\commentstrue

\ifcomments
\newcommand{\authornote}[3]{\textcolor{#1}{[#3 ---#2]}}
\newcommand{\todo}[1]{\textcolor{red}{[TODO: #1]}}
\else
\newcommand{\authornote}[3]{}
\newcommand{\todo}[1]{}
\fi

\newcommand{\wss}[1]{\authornote{blue}{SS}{#1}}
\newcommand{\ds}[1]{\authornote{red}{DS}{#1}}
\newcommand{\mj}[1]{\authornote{red}{MSN}{#1}}
\newcommand{\cm}[1]{\authornote{red}{CM}{#1}}
\newcommand{\mh}[1]{\authornote{red}{MH}{#1}}

% team members should be added for each team, like the following
% all comments left by the TAs or the instructor should be addressed
% by a corresponding comment from the Team

\newcommand{\tm}[1]{\authornote{magenta}{Team}{#1}}


\begin{document}

\begin{table}[hp]
\caption{Revision History} \label{TblRevisionHistory}
\begin{tabularx}{\textwidth}{llX}
\toprule
\textbf{Date} & \textbf{Developer(s)} & \textbf{Change}\\
\midrule
Sept 22 & Kelvin Lin & Completed fields for names and team information\\
Sept 23 & Kelvin Lin & Created first draft of problem statement\\
\bottomrule
\end{tabularx}
\end{table}

\newpage

\maketitle

Genetic algorithms (GA) are used to search for near-optimal solutions to a wide variety of problems with incomplete or imperfect information information. Applications of genetic algorithms include automated design, bioinformatics, economics and game theory, and training neural networks. Students and professionals from many different industries, with technological and non-technological backgrounds can benefit from using GA. However, despite the numerous applications of GA, there is a persistent lack of training resources for students and workers to learn the technology. 

In an informal survey of online resources conducted by Team 8 found that the GA training resources online typically contained one of two flaws: they either assume that the reader has extensive knowledge in mathematics, or they lack practical demonstrations to show the power of GA. 

The former flaw presents GA as a complex tool exclusively used by mathematicians and other mathematically inclined professions. Extensive mathematical jargon may also confuse novice learners who have a weak grasp of mathematical notation and mathematical theorems. Moreover, large mathematical explanations may intimidate some learners, discouraging them from pursuing an otherwise powerful tool for their work. 

The latter illustrates GA as a narrowly scoped tool used to solve specific problems. It fails to demonstrate the versatility of GA and how GA works to find the near-optimal solutions to problems without a clear optimal solution. Online resources in this category often use simple static examples, which hides the evolution process from the learner. This could reduce the learner's appreciation for GA, as they cannot see how the algorithm finds the near-optimal solution.

Ultimately, despite the versatility in using GA to find near-optimal solutions to problems in a variety of different industries, there continues to exist a lack of accessible training resources for students and professionals to learn about GA. A vast majority of current resources either assumes the learner has an extensive background in mathematics, or they fail to provide the learner with a board and engaging overview of the algorithm. This leaves many learners discouraged or impartial about the useful about the algorithm. Accordingly, there is a dire need for ways to teach both students and professionals about GA in a manner that is both relevant and applicable to their daily lives.

\end{document}