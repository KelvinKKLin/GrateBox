\documentclass{article}

\usepackage{tabularx}
\usepackage{booktabs}
\usepackage{color}
\title{SE 3XA3: Problem Statement\\GrateBox}

\author{Team 8, Grate
		\\ Kelvin Lin (linkk4)
		\\ Eric Chaput (chaputem)
		\\ Jin Liu (liu456)
}

\date{}

%% Comments

\usepackage{color}

\newif\ifcomments\commentstrue

\ifcomments
\newcommand{\authornote}[3]{\textcolor{#1}{[#3 ---#2]}}
\newcommand{\todo}[1]{\textcolor{red}{[TODO: #1]}}
\else
\newcommand{\authornote}[3]{}
\newcommand{\todo}[1]{}
\fi

\newcommand{\wss}[1]{\authornote{blue}{SS}{#1}}
\newcommand{\ds}[1]{\authornote{red}{DS}{#1}}
\newcommand{\mj}[1]{\authornote{red}{MSN}{#1}}
\newcommand{\cm}[1]{\authornote{red}{CM}{#1}}
\newcommand{\mh}[1]{\authornote{red}{MH}{#1}}

% team members should be added for each team, like the following
% all comments left by the TAs or the instructor should be addressed
% by a corresponding comment from the Team

\newcommand{\tm}[1]{\authornote{magenta}{Team}{#1}}


\begin{document}
\begin{table}[hp]
\color{blue}
\caption{Revision History} \label{TblRevisionHistory}
\begin{tabularx}{\textwidth}{llX}
\toprule
\textbf{Date} & \textbf{Developer(s)} & \textbf{Change}\\
\midrule
Sept 22 & Kelvin Lin & Completed fields for names and team information\\
Sept 23 & Kelvin Lin & Created first draft of problem statement\\
Sept 23 & Eric Chaput & Edited and expanded first draft\\
Sept 23 & Kelvin Lin and Eric Chaput& Formatted document\\
Sept 25 & Kelvin Lin & Minor modifications to second paragraph\\
Sept 30 & Kelvin Lin & Made stakeholders more explicit (Line 44)\\
\bottomrule
\end{tabularx}
\end{table}

\newpage
\color{black}
\maketitle

Genetic algorithms (GAs) search for near-optimal solutions to a wide variety of 
problems with incomplete or imperfect information by emulating the process of 
natural selection. Applications of genetic algorithms include automated design, 
bioinformatics, economics, game theory, and training neural networks. 
Stakeholders like students 
and professionals from many different industries, with technical and 
non-technical backgrounds can benefit from using GAs. However, despite the 
numerous applications of GAs, there is a persistent lack of online training 
resources for students and professionals to learn about them. 

An informal survey of online resources conducted by Team 8 found that the GA 
training resources typically contained one of two flaws: they either assumed 
that the reader has extensive knowledge in mathematics, or they fail to provide 
practical demonstrations. The former flaw presents GAs as a complex tool used 
exclusively by mathematically inclined professionals. Extensive mathematical 
jargon might confuse novice learners who have a weak grasp of mathematical 
notation and mathematical theorems. Moreover, large mathematical explanations 
could intimidate learners, discouraging them from pursuing an otherwise powerful 
tool for their work. The latter flaw illustrates GAs as a narrowly scoped tool 
used to solve specific problems. It fails to demonstrate the versatility of GAs 
and how GAs work to find the near-optimal solutions to problems without a clear 
optimal solution. Online resources in this category often use simple static 
examples, which hide the evolutionary process from the learner. This reduces the 
learner's appreciation for GAs, as they cannot see how the algorithms find the 
near-optimal solution. The key to learning is to keep the student engaged which 
current resources fail to do.

Ultimately, despite the versatility of GAs to find near-optimal solutions to 
problems in a variety of different industries, there continues to be a lack of 
accessible training resources for students and professionals to learn about GAs. 
This leaves many learners discouraged or impartial about the utility of these 
algorithms. Accordingly, there is a dire need for ways to teach both students 
and professionals about GAs in a manner that is both relevant and applicable to 
their daily lives.



\end{document}

