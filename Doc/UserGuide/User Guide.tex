\documentclass{article}

\usepackage{graphicx}
\usepackage{booktabs}
\usepackage{tabularx}
\usepackage{hyperref}
\usepackage{color}

\title{SE 3XA3: User Guide\\GrateBox}

\author{Team 8, Grate
		\\ Kelvin Lin (linkk4)
		\\ Eric Chaput (chaputem)
		\\ Jin Liu (liu456)
}

\date{}

%% Comments

\usepackage{color}

\newif\ifcomments\commentstrue

\ifcomments
\newcommand{\authornote}[3]{\textcolor{#1}{[#3 ---#2]}}
\newcommand{\todo}[1]{\textcolor{red}{[TODO: #1]}}
\else
\newcommand{\authornote}[3]{}
\newcommand{\todo}[1]{}
\fi

\newcommand{\wss}[1]{\authornote{blue}{SS}{#1}}
\newcommand{\ds}[1]{\authornote{red}{DS}{#1}}
\newcommand{\mj}[1]{\authornote{red}{MSN}{#1}}
\newcommand{\cm}[1]{\authornote{red}{CM}{#1}}
\newcommand{\mh}[1]{\authornote{red}{MH}{#1}}

% team members should be added for each team, like the following
% all comments left by the TAs or the instructor should be addressed
% by a corresponding comment from the Team

\newcommand{\tm}[1]{\authornote{magenta}{Team}{#1}}


\begin{document}

\newpage

\maketitle

\section{How to access GrateBox}

GrateBox is a web application created to educate and entertain users by the use of genetic algorithms. There are two ways to access and use the GrateBox application. The first method iss to navigate to the the url to access the online version of GrateBox. To do so, open your internet browser and enter the following url \href{http://ugweb.cas.mcmaster.ca/~linkk4/}{http://ugweb.cas.mcmaster.ca/~linkk4/}. GrateBox can also be run locally on your machine. To do this first navigate to to GrateBox's GitHub repository found at the following url \href{https://gitlab.cas.mcmaster.ca/linkk4/GrateBox/tree/master}{https://gitlab.cas.mcmaster.ca/linkk4/GrateBox/tree/master}. Then install the zip file by clicking the install button near the top right of the screen to download the zip file containing GrateBox. Extract the zip file to a location of your choosing and navigate to the src folder and double select the GrateBoxBootStrap.html file to open it.

\section {How to use Gratebox}

Once you access GrateBox, you will see a number of things All of GrateBox's functionality is contained on one screen. Under the simulation title, you will see a simulation of a randomly created generation of cars. By default, this simulation will generate an initial population of three cars for the first generation, with the best two cars of each generation selected for breeding the next generation, and a mutation rate of two percent. These values can be seen under the Parameters title, and can be edited by the user. To edit a parameter, simply enter a new value under the corresponding parameter title and click enter. The simulation will then be run from the start with the new values. Also under the Parameters title can be seen a pause button. To pause the simulation at any time click the pause button. To unpause the simulation, simply select the pause button again. Further down the page you will see text giving you information about the simulation and about GrateBox and Genetic Algorithms in general.

\section {Images for reference}



\end{document}
