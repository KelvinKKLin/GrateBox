\documentclass[12pt, titlepage]{article}

\usepackage{booktabs}
\usepackage{tabularx}
\usepackage{hyperref}
\hypersetup{
    colorlinks,
    citecolor=black,
    filecolor=black,
    linkcolor=red,
    urlcolor=blue
}

\title{SE 3XA3: Test Plan\\Genetic Cars}

\author{Team 8, Grate
		\\ Kelvin Lin (linkk4)
		\\ Eric Chaput (chaputem)
		\\ Jin Liu (liu456)
}


\usepackage[round]{natbib}

\date{\today}

%% Comments

\usepackage{color}

\newif\ifcomments\commentstrue

\ifcomments
\newcommand{\authornote}[3]{\textcolor{#1}{[#3 ---#2]}}
\newcommand{\todo}[1]{\textcolor{red}{[TODO: #1]}}
\else
\newcommand{\authornote}[3]{}
\newcommand{\todo}[1]{}
\fi

\newcommand{\wss}[1]{\authornote{blue}{SS}{#1}}
\newcommand{\ds}[1]{\authornote{red}{DS}{#1}}
\newcommand{\mj}[1]{\authornote{red}{MSN}{#1}}
\newcommand{\cm}[1]{\authornote{red}{CM}{#1}}
\newcommand{\mh}[1]{\authornote{red}{MH}{#1}}

% team members should be added for each team, like the following
% all comments left by the TAs or the instructor should be addressed
% by a corresponding comment from the Team

\newcommand{\tm}[1]{\authornote{magenta}{Team}{#1}}


\begin{document}

\maketitle

\pagenumbering{roman}
\tableofcontents
\listoftables
\listoffigures

\begin{table}[bph]
\caption{\bf Revision History}
\begin{tabularx}{\textwidth}{p{3cm}p{2cm}X}
\toprule {\bf Date} & {\bf Version} & {\bf Notes}\\
\midrule
Oct 24 & 1.0 & Imported Template and completed survey's and non-functional requirements  testing\\
Oct 25 & 1.1 & Implemented non-functional requirements testing\\
Oct 30 & 1.2 & First test case additions\\
Oct 31 & 1.3 & Second test case additions\\
Oct 31 & 1.4 & Final edits\\
\bottomrule
\end{tabularx}
\end{table}

\newpage

\pagenumbering{arabic}

\section{General Information}

\subsection{Purpose}

The purpose of this project's testing is to affirm that all requirements outlined in the Requirements Specifications document have been met and that the Genetic Cars software was implemented properly.

\subsection{Scope}

This test plan presents a basis for the testing of software functionality. It has the objectives of proving that the Genetic Cars project has met all the requirements outlined in the Requirements Specification document and of attaching metrics to those requirements for the sake of quantifying them. It also serves as a means to arrange testing activities. It will present what is to be tested and will act as an outline for testing methods and tools to be utilized.

\subsection{Acronyms, Abbreviations, and Symbols}
	
\begin{table}[hbp]
\caption{\textbf{Table of Abbreviations}} \label{Table}

\begin{tabularx}{\textwidth}{p{3cm}X}
\toprule
\textbf{Abbreviation} & \textbf{Definition} \\
\midrule
PoC & Proof of Concept\\
SRS & Software Requirements Specification\\
GC & Genetic Cars\\
GUI & Graphical User Interface\\
GA & Genetic Algorithms\\
\bottomrule
\end{tabularx}

\end{table}

\begin{table}[!htbp]
\caption{\textbf{Table of Definitions}} \label{Table}

\begin{tabularx}{\textwidth}{p{3cm}X}
\toprule
\textbf{Term} & \textbf{Definition}\\
\midrule
Structural Testing & Testing derived from the internal structure of the software\\
Functional Testing & Testing derived from a description of how the program functions (most often drawn from requirements)\\
Dynamic Testing & Testing which includes having test cases run during execution\\
Static Testing & Testing that does not involve program execution\\
Manual Testing & Testing conducted by people\\
Automated Testing & Testing that is run automatically by software\\
A majority of tested users & Defined as 80 percent of the users tested\\
\bottomrule
\end{tabularx}

\end{table}	

\subsection{Overview of Document}

The Genetic Cars project will re-implement the code of the open source project BoxCar2D. The software's requirements are outlined in the Requirements Specifications document.

\section{Plan}
	
\subsection{Software Description}

The software will allow users to model and learn about genetic algorithms in a fun and educational way. The implementtation will be completed in JavaScript.

\subsection{Test Team}

The individuals responsible for the testing of this project are  Kelvin Lin, Eric Chaput, and  Jin Liu. Jin Liu is in charge of testing for functional requirements. Eric Chaput is in charge of testing for non-functional requirements and for surveying representaative users for feedback and testing purposes. Kelvin Lin as team leader shall oversee time management for testing but will otherwise take no direct role in the testing process.

\subsection{Automated Testing Approach}

QUnit will be the primary tool used for testing this project. It will be used to automate unit testing. JavaScript also supports the principle of self-testing so many automated tests through JavaScript itself will also be viable.

\subsection{Testing Tools}

QUnit will be the primary tool used for testing this project. It will be used to automate unit testing. All group members are familiar with QUnit's Java equivalent, JUnit, and so minimal instruction in QUnit will be necessary. Testing will also be conducted in JavaScript itself as there are many testing methods native to JavaScript itself.

\subsection{Testing Schedule}
		
See Gantt Chart at the following url ...

\section{System Test Description}
	
\subsection{Tests for Functional Requirements}

\subsubsection{Genetic Algorithm}

\begin{enumerate}

\item{test-id1\\}

Type: Functional, Dynamic, Manual, Static etc.
					
Initial State: 
					
Input: 
					
Output: 
					
How test will be performed: 
					
\item{test-id2\\}

Type: Functional, Dynamic, Manual, Static etc.
					
Initial State: 
					
Input: 
					
Output: 
					
How test will be performed: 

\end{enumerate}

\subsubsection{Car Model}

\begin{enumerate}

\item{test-id1\\}

Type: Functional, Dynamic, Manual, Static etc.
					
Initial State: 
					
Input: 
					
Output: 
					
How test will be performed: 
					
\item{test-id2\\}

Type: Functional, Dynamic, Manual, Static etc.
					
Initial State: 
					
Input: 
					
Output: 
					
How test will be performed: 

\end{enumerate}

\subsubsection{Graphics}

\begin{enumerate}

\item{GR-1.1\\}

Type: Structural, Dynamic, Manual
					
Initial State: Nothing displayed on screen.
					
Input: Correct car created from car model. (i.e. a car object)
					
Output: Car created in graphics pane. Car's dimensions, values, etc. relative to numerical values given.
					
How test will be performed: Car values will be entered into graphics modules. Graphics modules will car given these values. Test is  successful if generated car appears as expected given numerical values (this is manually predetermined). Cars also compared to those in the original box car 2D application this project is based on.

\item{GR-1.2\\}

Type: Structural, Dynamic, Manual
					
Initial State: Nothing displayed on screen (possibly road as defined in GR-2).
					
Input: Incorrect car generaged from car model. (i.e. a car object where values are unacceptable, for example 9 vector magnitudes)
					
Output: Error message
					
How test will be performed: Car values will be entered into graphics modules. Graphics modules will car given these values. Test is successful if error message displayed.

\item{GR-1.3\\}

Type: Structural, Dynamic, Manual
					
Initial State: Nothing displayed on screen (possibly road as defined in GR-2).
					
Input: Car generated from car model with invalid values (i.e a negative vector magnitude)
					
Output: Error Message
					
How test will be performed: How test will be performed: Car values will be entered into graphics modules. Graphics modules will car given these values. Test is successful if error message displayed.
					
\item{GR-2.1\\}

Type: Structural, Dynamic, Manual
					
Initial State: Nothing displayed on screen.
					
Input:  Valid road algorithm.
					
Output: A graphical representation of a rode that is generated by the algorithm
					
How test will be performed: Road algorithm fed into graphical creation module. Generated road created by this equation. Test successful if given road corresponds to algorithm.

\item{GR-2.2\\}

Type: Structural, Dynamic, Manual
					
Initial State: Nothing displayed on screen.
					
Input: Given road algorithm invalid.
					
Output: Error Message
					
How test will be performed: Road algorithm fed into graphical creation module. Generated road created by this equation. Test successful if given road corresponds to algorithm.

\end{enumerate}

\subsubsection{Fitness and Score}

\begin{enumerate}

\item{FI-1\\}

Type: Structural, Static, Automatic
					
Initial State: Program installed onto system and launched or open in a web browser. Generation of cars entered into program and simulation run to determine fitness.
					
Input: Series of car locations from simulation (x,y coords as described in PoC demonstration)
					
Output: Final fitness values that correspond to those coordinates and car seeds.
					
How test will be performed: Fitness values from corresponding coordinates and car speeds from pre determined cars will be calculated outside of the program. The program will then be run with these pre determined cars to determine the accuracy of these values. 
					
\item{FI-2\\}

Type: Structural, Static, Automatic
					
Initial State: Program installed onto system and launched or open in a web browser. Generation of cars entered into program and simulation run to determine fitness.
					
Input: Series of fitness values from pre determined cars (See FI-1).
					
Output: Determination of highest fitness value from fitness values and display of said value.
					
How test will be performed:  Predetermined list of fitness values will be entered into the program. From these values the largest will determined (this value will have been caluclated seperately and entered into the unit test prior to this). If the value is the same the test is considered a success. A manual portion to this test will confirm that the value displays properly in the GUI.

\end{enumerate}

\subsubsection{Other GUI elements}

\begin{enumerate}

\item{GU-1\\}

Type: Structural, Dynamic, Manual
					
Initial State: Program installed onto system and launched or open in a web browser. Generation of cars entered into program and simulation running to determine fitness.
					
Input: Running of simulation.
					
Output: Movement of health bars in health bar GUI element relative to health of respective cars.
					
How test will be performed: Fitness determining simulation will be run. Health bars and how they move relative to respective cars will be observed. For this test specifically, a method will be written to display the numerical health values next to the health bars to determine the accuracy of these. The test will be considered a success if health bars graphically correspond to the numerical values of the respective cars. 
					
\item{GU-2\\}

Type: Structural, Dynamic, Automatic
					
Initial State: Program installed onto system and launched or open in a web browser. Generation of cars entered into program and simulation run to determine fitness.
					
Input: Running of simulation.
					
Output: Text file that contains numerical values with lables that will be displayed to the user (generation, cars alive, distance, height)
					
How test will be performed: A text file will be created based on the expected values of the generation, cars alive, distance, and height values after one minute with a particular random seed. This same seed will then be used in the program and the text of the above will be written to a text file after one minute of simulation. These text files will then be compared for testing purposes (i.e. are the values we expected the values we recieved). This will test for both the accuracy of these values and the accurate display of the values. The test will be considered a success if actual values match up 95 percent with what is expected.

\end{enumerate}

\subsection{Tests for Nonfunctional Requirements}

\subsubsection{Look and Feel}

\begin{enumerate}

\item{LF-1\\}

Type: Structural, Static, Manual
					
Initial State: Program installed onto system and launched or open in a web browser.
					
Input/Condition: Users asked to rate the visual asthetic of the program.
					
Output/Result: A majority of tested users shall agree that the visual aesthetic of the program is favorable.
					
How test will be performed: A test group of representative users (as defined in the development document) will be given two minutes of time to explore the program, its functions, and its outputs. This sample of users will then be asked to fill out a survey (see section 7.2) asking for their input. Test results shall be determined from those responses (i.e. if a majority of representative users rated the visual aesthetic of the program favorably then this test would be a success).
					
\item{LF-2\\}

Type: Type: Structural, Static, Manual
					
Initial State: Program installed onto system and launched or open in a web browser.
					
Input: Users asked to rate the style of the program
					
Output: A majority of tested users shall agree that the style of the program is favorable.
					
How test will be performed: A test group of representative users (as defined in the development document) will be given two minutes of time to explore the program, its functions, and its outputs. This sample of users will then be asked to fill out a survey (see section 7.2) asking for their input. Test results shall be determined from those responses (i.e. if a majority of representative users rated the style of the program favorably then this test would be a success).

\end{enumerate}

\subsubsection{Usability}

\begin{enumerate}

\item{US-1\\}

Type: Structural, Static, Manual
					
Initial State: Program installed onto system and launched or open in a web browser.
					
Input/Condition: Users given a list of tasks to accomplish.
					
Output/Result: A majority of tested users shall complete the tasks given two minutes.
					
How test will be performed: Users will be asked to accomplush the following:
- Run and install the program
- Identify the relevance of each of the elements of the GUI
- Demonstrate an understanding of genetic algorithms to a reasonable extent

\item{US-2\\}

Type: Type: Structural, Static, Manual
					
Initial State: Program files required to install the program are provided but uninstalled or web browser not yet directed to GC web page.
					
Input: Users asked to install the program or navigate to the web page given the url but without further assistance.
					
Output: A majority of tested users shall successfully install the program or navigate to the web page without assistance.
					
How test will be performed: A test group of representative users (as defined in the development document) will be given two minutes of time to install the program or navigate to the GC web page given the url. Program files will be downloaded to a users computer ony user request by testers if necessary or users will access uninstalled files on tester's devices.

\item{US-3\\}

Type: Type: Structural, Static, Manual
					
Initial State: Previous two tests (US-1 and US-2) conducted.
					
Input: Users asked to rate ease of installation and ease of use.
					
Output: A majority of tested users shall agree  that the program's useability is high.
					
How test will be performed: How test will be performed: A test group of representative users (as defined in the development document) will be given two minutes of time to explore the program, its functions, and its outputs. This sample of users will then be asked to fill out a survey (see section 7.2) asking for their input. Test results shall be determined from those responses (i.e. if a majority of representative users agree that the program's useability is high then this test will be considered a success).

\end{enumerate}

\subsubsection{Performance}

\begin{enumerate}

\item{PF-1\\}

Type: Structural, Dynamic, Automatic
					
Initial State: Program installed onto system and launched or open in a web browser.
					
Input/Condition: Program inititates one generation of genetic cars.
					
Output/Result: Generation created, displayed, and mutated within 20 seconds.
					
How test will be performed: Built in java script timer method will be used with Q-Unit (see section on unit testing) to record the time it takes for 100 generations. If all fall below 20 seconds from beggining to end then the test will be considered a success.


\item{PF-2\\}

Type: Type: Functional, Static, Manual
					
Initial State: Program installed onto system and launched or open in a web browser.
					
Input: Users asked to rate the speed of the program to the best of their ability.
					
Output: A majority of tested users shall agree that the speed of the program is favorable
					
 How test will be performed: A test group of representative users (as defined in the development document) will be given two minutes of time to explore the program, its functions, and its outputs. This sample of users will then be asked to fill out a survey (see section 7.2) asking for their input. Test results shall be determined from those responses (i.e. if a majority of representative users agree that the program's speed is favourable then this test will be considered a success, speed defined in the survey).

\item{PF-3\\}

Type: Structural, Dynamic, Automatic
					
Initial State: Program installed onto system and launched or open in a web browser.
					
Input/Condition: Program inititates one generation of genetic cars.
					
Output/Result: All numerical values accurate to what they should be.
					
How test will be performed: Unit testing through Q-Unit and native Java Script accuracy testing methods will be used to determine the validity of all numerical values and equations given. If all are valid then the test will be considered a success.


\item{PF-4\\}

Type: Type: Functional, Static, Manual
					
Initial State: Program installed onto system and launched or open in a web browser.
					
Input: Users asked to rate the accuracy of the program to the best of their ability.
					
Output: A majority of tested users shall agree that the accuracy of the program is favorable
					
 How test will be performed: A test group of representative users (as defined in the development document) will be given two minutes of time to explore the program, its functions, and its outputs. This sample of users will then be asked to fill out a survey (see section 7.2) asking for their input. Test results shall be determined from those responses (i.e. if a majority of representative users agree that the program's accuracy is favourable then this test will be considered a success, accuracy defined in the survey).

\end{enumerate}

\section{Tests for Proof of Concept}

Proof of Concept testing will be focused on verifying and validating the means by which automated testing will be performed. This will include automated testing of the genetic algorithm and then car model, as well as automated testing of graphical components to the extent possible.

\subsection{Genetic algorithm and car model}

\begin{enumerate}

\item{UC-1\\}

Type: Functional, Dynamic, Automated
					
Initial State: Program has been compiled but no additional inputs have been made.
					
Input: Creation of a car. Set random seed.
					
Output: Complete car object with random values that correspond with the given random seed. 100 percent match with estimated car and car values.
					
How test will be performed: Unit test will be set with a random seed and instructed to create 100 different cars given this seed. Final cars will then be compared to estimated cars. Test will be considered a success if generated cars are a 100 percent match to the 1st, 2nd, 3rd, 5th, 10th, and 50th cars estimated given the random seed. Test used to determine Q-Unit limitations on preset mathematical formulas.
					
\item{UC-2\\}

Type: Functional, Dynamic, Automated
					
Initial State: Program has created several cars as in test case UC-1 but no additional instructions have been given.
					
Input: Several pre made cars with non-random values.
					
Output: A new generation of cars created by mutating and crossing the generation of cars given.
					
How test will be performed: Mutation factor will be set to 10 percent. Final generation of cars will be estimated given the mutation factor and the pre generated cars. Final generation of cars will be generated using Q-Unit. Test will be considered a success if generated cars are a 100 precent match to the 1st, 2nd, 3rd, 5th, 10th, and 50th cars estimated. Test used to determine Q-Unit limitations on random generation.

\end{enumerate}

\subsection{Graphics}

\begin{enumerate}

\item{UG-1\\}

Type: Functional, Dynamic Automated
					
Initial State: Program open and running with generation of cars generated as in UC-1.
					
Input: Run simulation for the generation of cars. Predetermined graphical representation of simulation for comparison.
					
Output: Graphic representation of generation of cars for comparison to predetermined graphical representation.
					
How test will be performed: Generation of cars given will give estimated graphical output for 30 frames. Generation of cars given will be run through the program's graphics engine. 1st, 2nd, 12th, and 30th, frames. Comparison will be drawn using JavaScript in built image comparison. Test will be considered a success if 95 percent similarity reached. Test used to see Q-Unit limitations on a graphical level.

\end{enumerate}
	
\section{Comparison to Existing Implementation}	

There are is one test that compares the program to the Existing Implementation of the program please refer to:
- test GR 1.1 in Tests for Functional Requirements - Graphics
				
\section{Unit Testing Plan}

Unit testing will be conducted using the QUnit software outlined in the development document.
		
\subsection{Unit testing of internal functions}

In order to create unit tests for the internal functions of the program certain methods that return values
can be tested. This will involve taking the methods and giving them input values. Given what they are
supposed to output and that they actually output a series of unit tests can be created. Unit tests will include
tests that contain proper inputs and inputs that generate exceptions.  Anything that needs to be imported will already be done
by the individual classes. We will be using coverage metrics to determine how much of our code we have
covered. Our goal is to cover as much as possible in order to make sure that we test all functions adequately.
Our goal percentage to beat will be 85 percent.
		
\subsection{Unit testing of output files}		

Our program generates two primary outputs, the technical output generated by the genetic algorithms, and the graphical output displayed to the user. The genetic algorithms can be unit tested as explained above in the "Unit testing of internal functions" section. Graphical output will be harder to unit test, however Grate is looking into the prospect of pre generating expected graphical outcomes with given inputs and comparing those to the graphical outputs generated by the GC project. These images can then be compared by a unit test by percentage similarity. Currently the threshold for percentage similarity stands at 95 percent.

\bibliographystyle{plainnat}

\bibliography{SRS}

\newpage

\subsection{Usability Survey Questions?}

Note that many of the given survey questions may not pertain to a particular test case and are present in the survey so that users may give potentially valuable input that the testing team has not thought to request of them. This survey shall be reformated and placed on google docs at a later date.

1. How would you rate the "visual aesthetic" of this program? (i.e. Were the various elements of the user interface understandable, was it visually appealing)\\
- Favorable
- No opinion
-Unfavorable

2. How would you rate the "style" of this program? (i.e. Was the program professional enough, was it inviting enough,  do you feel you can you trust the product)\\
- Favorable
- No opinion
-Unfavorable

3. How would you rate the "usability" of this program? (i.e. Were the various functions obvious at first sight, did the program give the feedback you neeeded, was it a hassle to install/access online)\\
- Favorable
- No opinion
-Unfavorable

4. How would you rate the "speed" of the program? (i.e Did you have to wait while the program loaded or performed functions, did the program run smoothly throughout, did you experiance choppyness)\\
- Favorable
- No opinion
-Unfavorable

5. How would you rate the "accuracy" of the program? (i.e Did you feel the program displayed a valid interpretation of genetic algorithms based on your understanding of them, was the program mathematically sound from what you could see)\\
- Favorable
- No opinion
-Unfavorable

6. How would you rate your overall user experiance with the Genetic Cars application?\\
- Favorable
- No opinion
-Unfavorable

7. Did you find your experiance with the Genetic Cars application educational? (i.e. do you feel you undersant more about genetic algorithms as a result?)\\
-Yes
-Maybe
-No

8. Would you recommend the Genetic Cars application to a friend or relative?\\
-Yes
-Maybe
-No

9. Are there any suggestion or recomendations you would like to make towards the Grate development team to help us improve the application?\\
(USER INPUT)

\end{document}