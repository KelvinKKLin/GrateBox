\documentclass[12pt, titlepage]{article}

\usepackage{booktabs}
\usepackage{tabularx}
\usepackage{hyperref}
\hypersetup{
    colorlinks,
    citecolor=black,
    filecolor=black,
    linkcolor=red,
    urlcolor=blue
}

\title{SE 3XA3: Test Plan\\GrateBox}

\author{Team 8, Grate
		\\ Kelvin Lin (linkk4)
		\\ Eric Chaput (chaputem)
		\\ Jin Liu (liu456)
}


\usepackage[round]{natbib}

\date{\today}

%% Comments

\usepackage{color}

\newif\ifcomments\commentstrue

\ifcomments
\newcommand{\authornote}[3]{\textcolor{#1}{[#3 ---#2]}}
\newcommand{\todo}[1]{\textcolor{red}{[TODO: #1]}}
\else
\newcommand{\authornote}[3]{}
\newcommand{\todo}[1]{}
\fi

\newcommand{\wss}[1]{\authornote{blue}{SS}{#1}}
\newcommand{\ds}[1]{\authornote{red}{DS}{#1}}
\newcommand{\mj}[1]{\authornote{red}{MSN}{#1}}
\newcommand{\cm}[1]{\authornote{red}{CM}{#1}}
\newcommand{\mh}[1]{\authornote{red}{MH}{#1}}

% team members should be added for each team, like the following
% all comments left by the TAs or the instructor should be addressed
% by a corresponding comment from the Team

\newcommand{\tm}[1]{\authornote{magenta}{Team}{#1}}


\begin{document}

\maketitle

\pagenumbering{roman}
\tableofcontents
\listoftables
\listoffigures

\begin{table}[bph]
\caption{\bf Revision History}
\begin{tabularx}{\textwidth}{p{3cm}p{2cm}X}
\toprule {\bf Date} & {\bf Version} & {\bf Notes}\\
\midrule
Oct 24 & 1.0 & Imported Template and completed survey's and non-functional 
requirements  testing\\
Oct 25 & 1.1 & Implemented non-functional requirements testing\\
Oct 30 & 1.2 & First test case additions\\
Oct 31 & 1.3 & Second test case additions\\
Oct 31 & 1.4 & Final edits\\
Dec 07 & 1.5 & Final edits for Rev 1\\
\bottomrule
\end{tabularx}
\end{table}

\newpage

\pagenumbering{arabic}

\section{General Information}

\subsection{Purpose}

The purpose of this project's testing is to affirm that all requirements 
outlined in the Requirements Specifications document have been met and that the 
GrateBox software was implemented properly.

\subsection{Scope}

This test plan presents a basis for the testing of software functionality. It 
has the objectives of proving that the GrateBox project has met all the 
requirements outlined in the Requirements Specification document and of 
attaching metrics to those requirements for the sake of quantifying them. It 
also serves as a means to arrange testing activities. It will present what is to 
be tested and will act as an outline for testing methods and tools to be 
utilized.

\subsection{Acronyms, Abbreviations, and Symbols}
	
\begin{table}[hbp]
\caption{\textbf{Table of Abbreviations}} \label{Table}

\begin{tabularx}{\textwidth}{p{3cm}X}
\toprule
\textbf{Abbreviation} & \textbf{Definition} \\
\midrule
PoC & Proof of Concept\\
SRS & Software Requirements Specification\\
GC & Genetic Cars\\
GUI & Graphical User Interface\\
GA & Genetic Algorithms\\
\bottomrule
\end{tabularx}

\end{table}

\begin{table}[!htbp]
\caption{\textbf{Table of Definitions}} \label{Table}

\begin{tabularx}{\textwidth}{p{3cm}X}
\toprule
\textbf{Term} & \textbf{Definition}\\
\midrule
Structural Testing & Testing derived from the internal structure of the 
software\\
Functional Testing & Testing derived from a description of how the program 
functions (most often drawn from requirements)\\
Dynamic Testing & Testing which includes having test cases run during 
execution\\
Static Testing & Testing that does not involve program execution\\
Manual Testing & Testing conducted by people\\
Automated Testing & Testing that is run automatically by software\\
A majority of tested users & Defined as 70 percent of the users tested\\
\bottomrule
\end{tabularx}

\end{table}	

\subsection{Overview of Document}

The GrateBox project will re-implement the code of the open source project 
BoxCar2D. The software's requirements are outlined in the Requirements 
Specifications document.

\section{Plan}
	
\subsection{Software Description}

The software will allow users to model and learn about genetic algorithms in a 
fun and educational way. The implementation will be completed in JavaScript.

\subsection{Test Team}

The individuals responsible for the testing of this project are  Kelvin Lin, 
Eric Chaput, and  Jin Liu. Jin Liu is in charge of testing for functional 
requirements. Eric Chaput is in charge of testing for non-functional 
requirements and for surveying representative users for feedback and testing 
purposes. Kelvin Lin as team leader shall oversee time management for testing 
but will otherwise take no direct role in the testing process.

\subsection{Automated Testing Approach}

QUnit will be the primary tool used for testing this project. It will be used to 
automate unit testing. JavaScript also supports the principle of self-testing so 
many automated tests through JavaScript itself will also be viable.

\subsection{Testing Tools}

QUnit will be the primary tool used for testing this project. It will be used to 
automate unit testing. All group members are familiar with QUnit's Java 
equivalent, JUnit, and so minimal instruction in QUnit will be necessary. 
Testing will also be conducted in JavaScript itself as there are many testing 
methods native to JavaScript itself. Expected code coverage is 70 percent for this project. Team members are expected to know how to write and execute simples Qunit commands such as assert.equal. FireFox, Chrome, Oprah, and Internet explorer will also be tested to test for browser compatability.

\subsection{Testing Schedule}
See Gantt Chart at
\href{https://gitlab.cas.mcmaster.ca/linkk4/GrateBox/tree/master/ProjectSchedule}{the 
GitLab Repository}.

\section{System Test Description}
	
\subsection{Tests for Functional Requirements}

\subsubsection{Genetic Algorithm}

\begin{enumerate}

%1.0 = Mutations
\item{GA-1.1\\}

Type: Structural, Dynamic, Automated
					
Initial State: Cars with a known standardized chromosome. 
					
Input: The chromosome of the cars with a mutation rate of 0\%.
					
Output: The original chromosome of the car.
					
How test will be performed: Automated testing with QUnit will be used in order 
to create cars with a standardized known chromosome. The unit test will send the 
car model through the GA module's mutate function, passing a mutation rate of 
0\%, and it will assert that the chromosome going into the module is the same 
chromosome coming out of the module. This test will be repeated  a predetermined 
number of times in order to increase the confidence in the correctness of the 
module.
					
\item{GA-1.2\\}

Type: Structural, Dynamic, Automated
					
Initial State: Cars with a known standardized chromosome. 
					
Input: The chromosome of the cars with a mutation rate of 100\%.
					
Output: A chromosome that is completely modified.
					
How test will be performed: Automated testing with QUnit will be used in order 
to create cars with a standardized known chromosome. The unit test will send the 
car model through the GA module's mutate function, passing a mutation rate of 
100\%, and it will assert that every element in the output chromosome is 
different than the output from the input chromosome. This test will be repeated  
a predetermined number of times in order to increase the confidence in the 
correctness of the module.

\item{GA-1.3\\}

Type: Structural, Dynamic, Automated
					
Initial State: Cars with a known standardized chromosome. 
					
Input: The chromosome of the cars with a mutation rate of -1\%.
					
Output: The module should produce an error.
					
How test will be performed: Automated testing with QUnit will be used in order 
to create cars with a standardized known chromosome. The unit test will send the 
car model through the GA module's mutate function, passing a mutation rate of 
-1\%, and it will assert that an error is produced. This test will be repeated  
a predetermined number of times in order to increase the confidence in the 
correctness of the module.

%Crossover
\item{GA-2.1\\}

Type: Structural, Dynamic, Automated
					
Initial State: An array of cars with known length, and known chromosomes is 
created.
					
Input: An array of cars with length known chromosomes, and an integer that 
specifies that the top 3 cars should be used.
					
Output: An array of cars with the first three cars of the input array as the 
first 3 elements, and the length of the array is the same as the input array.
					
How test will be performed: Automated testing with QUnit will be used in order 
to create an array of cars with a standardized known chromosome. The unit test 
will send the cars array through the GA module's crossover function, passing a 
parameter of 3, and it will assert that the length of the array is the same as 
the input array and the first 3 cars of the output array is the same as the 
first 3 cars of the output array. This test will be repeated  a predetermined 
number of times in order to increase the confidence in the correctness of the 
module.

\item{GA-2.2\\}

Type: Structural, Dynamic, Manual
					
Initial State: An array of 3 cars with known chromosomes is created.
					
Input: An array of 3 cars with known chromosomes, and an integer parameter that 
specifies that the top 2 cars should be used.
					
Output: An array of cars with the first two cars of the input array, and a third 
car that is a derivative of the first two cars.
					
How test will be performed: This test will be manually tested by running a test 
driver that calls the crossover function. The output will be sent to a text 
file, where a tester will manually confirm that the chromosomes of the third car 
are indeed swapped. The tester will also verify that the output array has 3 
elements. This test will be repeated a predetermined number of times in order to 
increase the confidence in the correctness of the module.

\item{GA-2.3\\}

Type: Structural, Dynamic, Automated
					
Initial State: An array of cars with known length, and known chromosomes is 
created.
					
Input: An array of cars with length known chromosomes, and an integer that 
specifies that the top 1 cars should be used.
					
Output: An error.
					
How test will be performed: Automated testing with QUnit will be used in order 
to create an array of cars with a standardized known chromosome. The unit test 
will send the cars array through the GA module's crossover function, passing a 
parameter of 1, and it will assert that an error is produced. This test will be 
repeated  a predetermined number of times in order to increase the confidence in 
the correctness of the module.

\item{GA-2.4\\}

Type: Structural, Dynamic, Automated
					
Initial State: An array of cars with known length, and known chromosomes is 
created.
					
Input: An array of cars with length known chromosomes, and an integer that 
specifies that 0 cars should be used.
					
Output: An error.
					
How test will be performed: Automated testing with QUnit will be used in order 
to create an array of cars with a standardized known chromosome. The unit test 
will send the cars array through the GA module's crossover function, passing a 
parameter of 0, and it will assert that an error is produced. This test will be 
repeated  a predetermined number of times in order to increase the confidence in 
the correctness of the module.

\item{GA-2.5\\}

Type: Structural, Dynamic, Automated
					
Initial State: An array of cars with known length, and known chromosomes is 
created.
					
Input: An array of cars with length known chromosomes, and an integer that 
specifies that -1 cars should be used.
					
Output: An error.
					
How test will be performed: Automated testing with QUnit will be used in order 
to create an array of cars with a standardized known chromosome. The unit test 
will send the cars array through the GA module's crossover function, passing a 
parameter of -1, and it will assert that an error is produced. This test will be 
repeated  a predetermined number of times in order to increase the confidence in 
the correctness of the module.

%Selection
\item{GA-3.1\\}

Type: Structural, Dynamic, Automated
					
Initial State: An array of cars with known length, and known chromosomes is 
created.
					
Input: An array of cars with length known chromosomes, and an integer that 
specifies that the top 3 cars should be selected.
					
Output: An array of 3 cars with the highest fitness functions.
					
How test will be performed: Automated testing with QUnit will be used in order 
to create an array of cars with a standardized known chromosome. The unit test 
will send the cars array through the GA module's selection function, passing a 
parameter of 3. The unit test will also search for the top three cars using an 
external search module. The unit test will car the two values and assert that 
they are the same.


\item{GA-3.2\\}

Type: Structural, Dynamic, Automated
					
Initial State: An array of cars with known length \textit{n}, and known 
chromosomes is created.
					
Input: An array of cars with length known chromosomes, and an integer that 
specifies that the top  \textit{n}+1 cars should be selected.
					
Output: An error.
					
How test will be performed: Automated testing with QUnit will be used in order 
to create an array of cars with a standardized known chromosome. The unit test 
will send the cars array through the GA module's selection function, passing a 
parameter of \textit{n}+1. The unit test will assert that an error is thrown.


\item{GA-3.3\\}

Type: Structural, Dynamic, Automated
					
Initial State: An array of cars with known length \textit{n}, and known 
chromosomes is created.
					
Input: An array of cars with length known chromosomes, and an integer that 
specifies that the top  0 cars should be selected.
					
Output: An error
					
How test will be performed: Automated testing with QUnit will be used in order 
to create an array of cars with a standardized known chromosome. The unit test 
will send the cars array through the GA module's selection function, passing a 
parameter of 0. The unit test will assert that an error is thrown.


\item{GA-3.4\\}

Type: Structural, Dynamic, Automated
					
Initial State: An array of cars with known length \textit{n}, and known 
chromosomes is created.
					
Input: An array of cars with length known chromosomes, and an integer that 
specifies that the top  -1 cars should be selected.
					
Output: An array of 3 cars with the highest fitness functions.
					
How test will be performed: Automated testing with QUnit will be used in order 
to create an array of cars with a standardized known chromosome. The unit test 
will send the cars array through the GA module's selection function, passing a 
parameter of -1. The unit test will assert that an error is thrown.

\end{enumerate}

\subsubsection{Car Model}

\begin{enumerate}

\item{CM-1.1\\}

Type: Structural, Dynamic, Automated
					
Initial State: No car generated.
					
Input: Pre determined random seed and valid car generation module run.
					
\textcolor{blue}{Output: Set of 2 wheels for a car.}
					
\textcolor{blue}{How test will be performed: Car will be randomly created. Number of wheels will 
be found to be 2 for a successful test. Note that the output 
should be the same number of wheels as there are wheel radii as in 1.2.}

\item{CM-1.2\\}

Type: Structural, Dynamic, Automated
					
Initial State: No car generated.
					
Input: Pre determined random seed and valid  car generation module run.
					
Output: Set of wheel radiuses for a car.
					
\textcolor{blue}{How test will be performed: Car will be randomly created. All wheel radii
will be found to be positive numbers within a certain range to be determined at 
a later date. Note that the output should be the same number of radii as 
there are wheels as in 1.1.}

\item{CM-1.3\\}

Type: Structural, Dynamic, Automated
					
Initial State: No car generated.
					
Input: Pre determined random seed and valid  car generation module run.
					
\textcolor{blue}{Output: Set of eight vertex positions for a car.}
					
\textcolor{blue}{How test will be performed: Car will be randomly created. All vertex positions 
will be found to be positions in 2-D space within a certain area.}

\item{CM-1.4\\}

Type: Structural, Dynamic, Automated
					
Initial State: No car generated
					
Input: Pre determined random seed and valid car generation module run.
					
\textcolor{blue}{Output: Set of eight vector angles for a car.}
					
\textcolor{blue}{How test will be performed: Car will be randomly created. All vertex angles will 
be found to be positive values within a certain range (within 2*pi)}

\item{CM-1.5\\}

Type: Structural, Dynamic, Automated
					
Initial State: No car generated
					
Input: Pre determined random seed and faulty car generation module run for wheel 
number.
					
Output: Error message displayed
					
How test will be performed:  Car will be randomly created. Error message should 
occur if test successful.

\item{CM-1.6\\}

Type: Structural, Dynamic, Automated
					
Initial State: No car generated
					
Input: Pre determined random seed and faulty car generation module run for wheel 
radiuses.
					
Output: Error message displayed
					
How test will be performed: Car will be randomly created. Error message should 
occur if test successful.

\item{CM-1.7\\}

Type: Structural, Dynamic, Automated
					
Initial State: No car generated
					
Input: Pre determined random seed and faulty car generation module run for 
vector angles.
					
Output: Error message displayed
					
How test will be performed: Car will be randomly created. Error message should 
occur if test successful.

\item{CM-1.8\\}

Type: Structural, Dynamic, Automated
					
Initial State: No car generated
					
\textcolor{blue}{Input: Pre determined random seed and faulty car generation module run for 
vertex number.}
					
Output: Error message displayed
					
\textcolor{blue}{How test will be performed: Car will be randomly created. Error message should 
occur if test successful.}
					
\item{CM-2\\}

Type: Structural, Dynamic, Manual
					
Initial State: Car generated with values created as in CM-1.
					
Input: Command to create car chromosome.
					
Output: Text file containing all car values in format for car chromosome.
					
How test will be performed:  Given car will be run through chromosome script. 
Output text file will be manually examined by testing team to determine 
validity. Chromesome will be successfully created if resulting text file 
contains the five corresponding arrays for a car and a mass value.
\end{enumerate}

\subsubsection{Graphics}

\begin{enumerate}

\item{GR-1.1\\}

Type: Structural, Dynamic, Manual
					
Initial State: Nothing displayed on screen.
					
Input: Correct car created from car model. (i.e. a car object)
					
Output: Car created in graphics pane. Car's dimensions, values, etc. relative to 
numerical values given.
					
How test will be performed: Car values will be entered into graphics modules. 
Graphics modules will create car given these values. Test is successful if generated 
car appears as expected given numerical values (this is manually predetermined). 
Cars also compared to those in the original BoxCar-2D application this project 
is based on.

\item{GR-1.2\\}

Type: Structural, Dynamic, Manual
					
Initial State: Nothing displayed on screen (possibly road as defined in GR-2).
					
Input: Incorrect car generated from car model. (i.e. a car object where values 
are unacceptable, for example 9 vector magnitudes)
					
Output: Error message
					
How test will be performed: Car values will be entered into graphics modules. 
Graphics modules will car given these values. Test is successful if error 
message displayed.

\item{GR-1.3\\}

Type: Structural, Dynamic, Manual
					
Initial State: Nothing displayed on screen (possibly road as defined in GR-2).
					
Input: Car generated from car model with invalid values (i.e a negative vector 
magnitude)
					
Output: Error Message
					
How test will be performed: How test will be performed: Car values will be 
entered into graphics modules. Graphics modules will car given these values. 
Test is successful if error message displayed.
					
\item{GR-2.1\\}

Type: Structural, Dynamic, Manual
					
Initial State: Nothing displayed on screen.
					
Input:  Valid road algorithm.
					
Output: A graphical representation of a rode that is generated by the algorithm
					
How test will be performed: Road algorithm fed into graphical creation module. 
Generated road created by this equation. Test successful if given road 
corresponds to algorithm.

\item{GR-2.2\\}

Type: Structural, Dynamic, Manual
					
Initial State: Nothing displayed on screen.
					
Input: Given road algorithm invalid.
					
Output: Error Message
					
How test will be performed: Road algorithm fed into graphical creation module. 
Test successful if no road generated and error message displayed.

\end{enumerate}

\subsubsection{Fitness and Score}

\begin{enumerate}

\item{FI-1\\}

Type: Structural, Static, Automatic
					
Initial State: Program installed onto system and launched or open in a web 
browser. Generation of cars entered into program and simulation run to determine 
fitness.
					
Input: Series of car locations from simulation (x,y coordinates as described in PoC 
demonstration)
					
Output: Final fitness values that correspond to those coordinates and car seeds.
					
How test will be performed: Fitness values from corresponding coordinates and 
car speeds from pre determined cars will be calculated outside of the program. 
The program will then be run with these pre determined cars to determine the 
accuracy of these values. 
					
\item{FI-2\\}

Type: Structural, Static, Automatic
					
Initial State: Program installed onto system and launched or open in a web 
browser. Generation of cars entered into program and simulation run to determine 
fitness.
					
Input: Series of fitness values from pre determined cars (See FI-1).
					
Output: Determination of highest fitness value from fitness values and display 
of said value.
					
How test will be performed:  Predetermined list of fitness values will be 
entered into the program. From these values the largest will determined (this 
value will have been calculated separately and entered into the unit test prior 
to this). If the value is the same the test is considered a success. A manual 
portion to this test will confirm that the value displays properly in the GUI.

\end{enumerate}

\subsubsection{Other GUI elements}

\begin{enumerate}

\item{GU-1\\}

Type: Structural, Dynamic, Manual
					
Initial State: Program installed onto system and launched or open in a web 
browser. Generation of cars entered into program and simulation running to 
determine fitness.
					
Input: Running of simulation.
					
Output: Movement of health bars in health bar GUI element relative to health of 
respective cars.
					
How test will be performed: Fitness determining simulation will be run. Health 
bars and how they move relative to respective cars will be observed. For this 
test specifically, a method will be written to display the numerical health 
values next to the health bars to determine the accuracy of these. The test will 
be considered a success if health bars graphically correspond to the numerical 
values of the respective cars. 
					
\item{GU-2\\}

Type: Structural, Dynamic, Automatic
					
Initial State: Program installed onto system and launched or open in a web 
browser. Generation of cars entered into program and simulation run to determine 
fitness.
					
Input: Running of simulation.
					
Output: Text file that contains numerical values with labels that will be 
displayed to the user (generation, cars alive, distance, height)
					
How test will be performed: A text file will be created based on the expected 
values of the generation, cars alive, distance, and height values after one 
minute with a particular random seed. This same seed will then be used in the 
program and the text of the above will be written to a text file after one 
minute of simulation. These text files will then be compared for testing 
purposes (i.e. are the values we expected the values we received). This will 
test for both the accuracy of these values and the accurate display of the 
values. The test will be considered a success if actual values match up 95 
percent with what is expected.

\end{enumerate}

\subsection{Tests for Nonfunctional Requirements}

\subsubsection{Look and Feel}

\begin{enumerate}

\item{LF-1\\}

Type: Structural, Static, Manual
					
Initial State: Program installed onto system and launched or open in a web 
browser.
					
Input/Condition: Users asked to rate the visual aesthetic of the program.
					
Output/Result: A majority of tested users shall agree that the visual aesthetic 
of the program is favourable.
					
How test will be performed: A test group of representative users (as defined in 
the development document) will be given two minutes of time to explore the 
program, its functions, and its outputs. This sample of users will then be asked 
to fill out a survey (see section 7.2) asking for their input. Test results 
shall be determined from those responses (i.e. if a majority of representative 
users rated the visual aesthetic of the program favourably then this test would 
be a success).
					
\item{LF-2\\}

Type: Type: Structural, Static, Manual
					
Initial State: Program installed onto system and launched or open in a web 
browser.
					
Input: Users asked to rate the style of the program
					
Output: A majority of tested users shall agree that the style of the program is 
favourable.
					
How test will be performed: A test group of representative users (as defined in 
the development document) will be given two minutes of time to explore the 
program, its functions, and its outputs. This sample of users will then be asked 
to fill out a survey (see section 7.2) asking for their input. Test results 
shall be determined from those responses (i.e. if a majority of representative 
users rated the style of the program favourably then this test would be a 
success).

\end{enumerate}

\subsubsection{Usability}

\begin{enumerate}

\item{US-1\\}

Type: Structural, Static, Manual
					
Initial State: Program installed onto system and launched or open in a web 
browser.
					
Input/Condition: Users given a list of tasks to accomplish.
					
Output/Result: A majority of tested users shall complete the tasks given two 
minutes.
					
How test will be performed: Users will be asked to accomplish the following:
- Run and install the program
- Identify the relevance of each of the elements of the GUI
- Demonstrate an understanding of genetic algorithms to a reasonable extent

\item{US-2\\}

Type: Type: Structural, Static, Manual
					
Initial State: Program files required to install the program are provided but 
uninstalled or web browser not yet directed to GC web page.
					
Input: Users asked to install the program or navigate to the web page given the 
url but without further assistance.
					
Output: A majority of tested users shall successfully install the program or 
navigate to the web page without assistance.
					
How test will be performed: A test group of representative users (as defined in 
the development document) will be given two minutes of time to install the 
program or navigate to the GrateBox web page given the url. Program files will be 
downloaded to a users computer ony user request by testers if necessary or users 
will access uninstalled files on tester's devices.

\item{US-3\\}

Type: Type: Structural, Static, Manual
					
Initial State: Previous two tests (US-1 and US-2) conducted.
					
Input: Users asked to rate ease of installation and ease of use.
					
Output: A majority of tested users shall agree  that the program's usability is 
high.
					
How test will be performed: How test will be performed: A test group of 
representative users (as defined in the development document) will be given two 
minutes of time to explore the program, its functions, and its outputs. This 
sample of users will then be asked to fill out a survey (see section 7.2) asking 
for their input. Test results shall be determined from those responses (i.e. if 
a majority of representative users agree that the program's usability is high 
then this test will be considered a success).

\end{enumerate}

\subsubsection{Performance}

\begin{enumerate}

\item{PF-1\\}

Type: Structural, Dynamic, Automatic
					
Initial State: Program installed onto system and launched or open in a web 
browser.
					
Input/Condition: Program initiates one generation of genetic cars.
					
Output/Result: Generation created, displayed, and mutated within 20 seconds.
					
How test will be performed: Built in JavaScript timer method will be used with 
Q-Unit (see section on unit testing) to record the time it takes for a
generation. If all fall below 20 seconds from beginning to end then the test 
will be considered a success.


\item{PF-2\\}

Type: Type: Functional, Static, Manual
					
Initial State: Program installed onto system and launched or open in a web 
browser.
					
Input: Users asked to rate the speed of the program to the best of their 
ability.
					
Output: A majority of tested users shall agree that the speed of the program is 
favourable
					
 How test will be performed: A test group of representative users (as defined in 
the development document) will be given two minutes of time to explore the 
program, its functions, and its outputs. This sample of users will then be asked 
to fill out a survey (see section 7.2) asking for their input. Test results 
shall be determined from those responses (i.e. if a majority of representative 
users agree that the program's speed is favourable then this test will be 
considered a success, speed defined in the survey).

\item{PF-3\\}

Type: Structural, Dynamic, Automatic
					
Initial State: Program installed onto system and launched or open in a web 
browser.
					
Input/Condition: Program initiates one generation of genetic cars.
					
Output/Result: All numerical values accurate to what they should be.
					
How test will be performed: Unit testing through Q-Unit and native Java Script 
accuracy testing methods will be used to determine the validity of all numerical 
values and equations given. If all are valid then the test will be considered a 
success.


\item{PF-4\\}

Type: Type: Functional, Static, Manual
					
Initial State: Program installed onto system and launched or open in a web 
browser.
					
Input: Users asked to rate the accuracy of the program to the best of their 
ability.
					
Output: A majority of tested users shall agree that the accuracy of the program 
is favourable
					
 How test will be performed: A test group of representative users (as defined in 
the development document) will be given two minutes of time to explore the 
program, its functions, and its outputs. This sample of users will then be asked 
to fill out a survey (see section 7.2) asking for their input. Test results 
shall be determined from those responses (i.e. if a majority of representative 
users agree that the program's accuracy is favourable then this test will be 
considered a success, accuracy defined in the survey).

\end{enumerate}

\section{Tests for Proof of Concept}

Proof of Concept testing will be focused on verifying and validating the means 
by which automated testing will be performed. This will include automated 
testing of the genetic algorithm and then car model, as well as automated 
testing of graphical components to the extent possible.

\subsection{Genetic algorithm and car model}

\begin{enumerate}

\item{UC-1\\}

Type: Functional, Dynamic, Automated
					
Initial State: Program has been run but no additional inputs have been made.
					
Input: Creation of a car. Set random seed.
					
Output: Complete car object with random values that correspond with the given 
random seed. 100 percent match with estimated car and car values.
					
How test will be performed: Unit test will be set with a random seed and 
instructed to create 100 different cars given this seed. Final cars will then be 
compared to estimated cars. Test will be considered a success if generated cars 
are a 100 percent match to the 1st, 2nd, 3rd, 5th, 10th, and 50th cars estimated 
given the random seed. Test used to determine Q-Unit limitations on preset 
mathematical formulas.
					
\item{UC-2\\}

Type: Functional, Dynamic, Automated
					
Initial State: Program has created several cars as in test case UC-1 but no 
additional instructions have been given.
					
Input: Several pre made cars with non-random values.
					
Output: A new generation of cars created by mutating and crossing the generation 
of cars given.
					
How test will be performed: Mutation factor will be set to 10 percent. Final 
generation of cars will be estimated given the mutation factor and the pre 
generated cars. Final generation of cars will be generated using Q-Unit. Test 
will be considered a success if generated cars are a 100 percent match to the 
1st, 2nd, 3rd, 5th, 10th, and 50th cars estimated. Test used to determine Q-Unit 
limitations on random generation.

\end{enumerate}

\subsection{Graphics}

\begin{enumerate}

\item{UG-1\\}

Type: Functional, Dynamic Automated
					
Initial State: Program open and running with generation of cars generated as in 
UC-1.
					
Input: Run simulation for the generation of cars. Predetermined graphical 
representation of simulation for comparison.
					
Output: Graphic representation of generation of cars for comparison to 
predetermined graphical representation.
					
How test will be performed: Generation of cars given will give estimated 
graphical output for 30 frames. Generation of cars given will be run through the 
program's graphics engine. 1st, 2nd, 12th, and 30th, frames. Comparison will be 
drawn using JavaScript in built image comparison. Test will be considered a 
success if 95 percent similarity reached. Test used to see Q-Unit limitations on 
a graphical level.

\end{enumerate}
	
\section{Comparison to Existing Implementation}	
There are is one test that compares the program to the Existing Implementation 
of the program please refer to test GR 1.1 in Tests for Functional Requirements 
- Graphics. Non-functional comparison is to be conducted through the user survey.
				
\section{Unit Testing Plan}

Unit testing will be conducted using the QUnit software outlined in the 
development document.
		
\subsection{Unit testing of internal functions}

In order to create unit tests for the internal functions of the program certain 
methods that return values
can be tested. This will involve taking the methods and giving them input 
values. Given what they are
supposed to output and that they actually output a series of unit tests can be 
created. Unit tests will include
tests that contain proper inputs and inputs that generate exceptions.  Anything 
that needs to be imported will already be done
by the individual classes. We will be using coverage metrics to determine how 
much of our code we have
covered. Our goal is to cover as much as possible in order to make sure that we 
test all functions adequately.
Our goal percentage to beat will be 85 percent.
		
\subsection{Unit testing of output files}		

Our program generates two primary outputs, the technical output generated by the 
genetic algorithms, and the graphical output displayed to the user. The genetic 
algorithms can be unit tested as explained above in the "Unit testing of 
internal functions" section. Graphical output will be harder to unit test, 
however Grate is looking into the prospect of pre generating expected graphical 
outcomes with given inputs and comparing those to the graphical outputs 
generated by the GC project. These images can then be compared by a unit test by 
percentage similarity. Currently the threshold for percentage similarity stands 
at 95 percent.

\newpage

\subsection{Usability Survey Questions?}
Note that many of the given survey questions may not pertain to a particular 
test case and are present in the survey so that users may give potentially 
valuable input that the testing team has not thought to request of them. This 
survey shall be reformatted and placed on Google Docs at a later date.

1. How would you rate the "visual aesthetic" of this program? (i.e. Were the 
various elements of the user interface understandable, was it visually 
appealing)

- Favourable

- No opinion

- Unfavourable

2. How would you rate the "style" of this program? (i.e. Was the program 
professional enough, was it inviting enough,  do you feel you can you trust the 
product)

- Favourable

- No opinion

- Unfavourable

3. How would you rate the "usability" of this program? (i.e. Were the various 
functions obvious at first sight, did the program give the feedback you needed, 
was it a hassle to install/access online)
- Favourable

- No opinion

- Unfavourable

4. How would you rate the "speed" of the program? (i.e Did you have to wait 
while the program loaded or performed functions, did the program run smoothly 
throughout, did you experience choppiness)

- Favourable

- No opinion

- Unfavourable

5. How would you rate the "accuracy" of the program? (i.e Did you feel the 
program displayed a valid interpretation of genetic algorithms based on your 
understanding of them, was the program mathematically sound from what you could 
see)

- Favourable

- No opinion

- Unfavourable

6. How would you rate your overall user experiance with the GrateBox
application?

- Favorable

- No opinion

-Unfavorable

7. Did you find your experiance with the GrateBox application educational? 
(i.e. do you feel you undersant more about genetic algorithms as a result?)

-Yes

-Maybe

-No

8. Would you recommend the GrateBox application to a friend or relative?

-Yes

-Maybe

-No

9. Are there any suggestion or recomendations you would like to make towards the 
Grate development team to help us improve the application?

(USER INPUT)

\textcolor{blue}{10. Did you feel like you were able to learn how GrateBox works without too much difficulty?}

\textcolor{blue}{-Yes}

\textcolor{blue}{-Maybe}

\textcolor{blue}{-No}

\end{document}
