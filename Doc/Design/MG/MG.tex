\documentclass[12pt, titlepage]{article}

\usepackage{fullpage}
\usepackage[round]{natbib}
\usepackage{multirow}
\usepackage{booktabs}
\usepackage{tabularx}
\usepackage{graphicx}
\usepackage{float}
\usepackage{hyperref}
\hypersetup{
    colorlinks,
    citecolor=black,
    filecolor=black,
    linkcolor=red,
    urlcolor=blue
}
\usepackage[round]{natbib}

\newcounter{acnum}
\newcommand{\actheacnum}{AC\theacnum}
\newcommand{\acref}[1]{AC\ref{#1}}

\newcounter{ucnum}
\newcommand{\uctheucnum}{UC\theucnum}
\newcommand{\uref}[1]{UC\ref{#1}}

\newcounter{mnum}
\newcommand{\mthemnum}{M\themnum}
\newcommand{\mref}[1]{M\ref{#1}}

\title{SE 3XA3: Module Guide\\Genetic Cars}

\author{Team 8, Grate
		\\ Kelvin Lin (linkk4)
		\\ Eric Chaput (chaputem)
		\\ Jin Liu (liu456)
}

\date{\today}

%% Comments

\usepackage{color}

\newif\ifcomments\commentstrue

\ifcomments
\newcommand{\authornote}[3]{\textcolor{#1}{[#3 ---#2]}}
\newcommand{\todo}[1]{\textcolor{red}{[TODO: #1]}}
\else
\newcommand{\authornote}[3]{}
\newcommand{\todo}[1]{}
\fi

\newcommand{\wss}[1]{\authornote{blue}{SS}{#1}}
\newcommand{\ds}[1]{\authornote{red}{DS}{#1}}
\newcommand{\mj}[1]{\authornote{red}{MSN}{#1}}
\newcommand{\cm}[1]{\authornote{red}{CM}{#1}}
\newcommand{\mh}[1]{\authornote{red}{MH}{#1}}

% team members should be added for each team, like the following
% all comments left by the TAs or the instructor should be addressed
% by a corresponding comment from the Team

\newcommand{\tm}[1]{\authornote{magenta}{Team}{#1}}


\begin{document}

\maketitle

\pagenumbering{roman}
\tableofcontents
\listoftables
\listoffigures

\begin{table}[bp]
\caption{\bf Revision History}
\begin{tabularx}{\textwidth}{p{3cm}p{2cm}X}
\toprule {\bf Date} & {\bf Version} & {\bf Notes}\\
\midrule
Nov 06 & 1.0 & Creation of template and first additions\\
Nov 08 & 1.1 & Added all elements not directly reliant on specific module names/uses\\
Nov 11 & 1.2 & All but module detail complete\\
Nov 13 & 1.3 & Final draft and editing complete\\
\bottomrule
\end{tabularx}
\end{table}

\newpage

\pagenumbering{arabic}

\section{Introduction}

Decomposing a system into modules is a commonly accepted approach to developing
software.  A module is a work assignment for a programmer or programming
team~\citep{ParnasEtAl1984}.  We are using decomposition
based on the principle of information hiding~\citep{Parnas1972a}.  This
principle supports design for change, because the ``secrets'' that each module
hides represent likely future changes.  

Our design follows the rules layed out by \citet{ParnasEtAl1984}, as follows:
\begin{itemize}
\item System details that are likely to change independently should be the
  secrets of separate modules.
\item Each data structure is used in only one module.
\item Any other program that requires information stored in a module's data
  structures must obtain it by calling access programs belonging to that module.
\end{itemize}

This MG specifies the modular structure of the system and is intended to allow both
designers and maintainers to easily identify the parts of the software.  The
potential readers of this document are as follows:

\begin{itemize}
\item New project members: This document can be a guide for a new project member
  to easily understand the overall structure and quickly find the
  relevant modules they are searching for.
\item Maintainers: The hierarchical structure of the module guide improves the
  maintainers' understanding when they need to make changes to the system. It is
  important for a maintainer to update the relevant sections of the document
  after changes have been made.
\item Designers: Once the module guide has been written, it can be used to
  check for consistency, feasibility and flexibility. Designers can verify the
  system in various ways, such as consistency among modules, feasibility of the
  decomposition, and flexibility of the design.
\end{itemize}

The rest of the document is organized as follows. Section
\ref{SecChange} lists the anticipated and unlikely changes of the software
requirements. Section \ref{SecMH} summarizes the module decomposition that
was constructed according to the likely changes. Section \ref{SecConnection}
specifies the connections between the software requirements and the
modules. Section \ref{SecMD} gives a detailed description of the
modules. Section \ref{SecTM} includes two traceability matrices. One checks
the completeness of the design against the requirements provided in the SRS. The
other shows the relation between anticipated changes and the modules. Section
\ref{SecUse} describes the use relation between modules.

\section{Anticipated and Unlikely Changes} \label{SecChange}

This section lists possible changes to the system. According to the likeliness
of the change, the possible changes are classified into two
categories. Anticipated changes are listed in Section \ref{SecAchange}, and
unlikely changes are listed in Section \ref{SecUchange}.

\subsection{Anticipated Changes} \label{SecAchange}

\begin{description}
\item[\refstepcounter{acnum} \actheacnum \label{acWebsite}:] The changes that may arrise from the placement of the Genetic Cars project online (i.e. web hosting choices, website layout, etc.) 
\item[\refstepcounter{acnum} \actheacnum \label{acUserInput}:] The ability for the user to alter inputs (i.e. the mutation rate, parameters for cars (vertex number etc.), parameters for the enviroment (gravity etc.)) 
\item[\refstepcounter{acnum} \actheacnum \label{acRoad}:] Changes to the overall structure of the road (i.e. using different mathematical forumals to generate it, more or less steep overall, length)
\item[\refstepcounter{acnum} \actheacnum \label{acAesthetics}:] Changes to the aesthetics of the program (i.e. display window shape and size, use of color, etc.)
\item[\refstepcounter{acnum} \actheacnum \label{acGeneticAlgorithm}:] Changes to the genetic algorithm (i.e. isolate for fastest car instead of farthest traveled, different means of reproduction, etc.)
\item[\refstepcounter{acnum} \actheacnum \label{acInitialFactors}:] Changes to the initial state of the enviroment and car population (i.e. larger/smaller initial population sizes)
\end{description}

\subsection{Unlikely Changes} \label{SecUchange}

\begin{description}
\item[\refstepcounter{ucnum} \uctheucnum \label{ucVehicleChange}:] Changes to the structure of the vehicle altogether (i.e. mabey it is no longer cars but vehicles with 3+ wheels or no wheels at all)
\item[\refstepcounter{ucnum} \uctheucnum \label{ucFitnessRequirements}:] Changes to the goal of fitness isolation (i.e. isolating for the worst vehicles instead of the best)
\item[\refstepcounter{ucnum} \uctheucnum \label{ucCarSpeed}:] Changes to the speed of componenets in simulations (i.e. the set speed of the wheels)
\item[\refstepcounter{ucnum} \uctheucnum \label{ucLibrary}:] Changes to the library used to generate cars (i.e. changes to the library that result in rounder or more abstract cars)
\end{description}

\section{Module Hierarchy} \label{SecMH}

This section provides an overview of the module design. Modules are summarized
in a hierarchy decomposed by secrets in Table \ref{TblMH}. The modules listed
below, which are leaves in the hierarchy tree, are the modules that will
actually be implemented.

\begin{description}
\item [\refstepcounter{mnum} \mthemnum \label{mHardware}:] Hardware hiding module
\item [\refstepcounter{mnum} \mthemnum \label{mCreateCar}:]  Car creation module
\item [\refstepcounter{mnum} \mthemnum \label{mEvolveCar}:] Evolve car module
\item [\refstepcounter{mnum} \mthemnum \label{mCreateRoad}:] Road creation module
\item [\refstepcounter{mnum} \mthemnum \label{mGraphicsDisplay}:] Graphics display module
\item [\refstepcounter{mnum} \mthemnum \label{mGeneticAlgorithm}:] Genetic Algorithm module
\item [\refstepcounter{mnum} \mthemnum \label{mRandomSeed}:] Random seed generation and manipulation module
\item [\refstepcounter{mnum} \mthemnum \label{mFitness}:] Fitness determination module
\item [\refstepcounter{mnum} \mthemnum \label{mSortingAndSearching}:] Sorting and Searching algorithms module
\item [\refstepcounter{mnum} \mthemnum \label{mPopulationGeneration}:] Population generation algorithms module
\end{description}


\begin{table}[h!]
\centering
\begin{tabular}{p{0.3\textwidth} p{0.6\textwidth}}
\toprule
\textbf{Level 1} & \textbf{Level 2}\\
\midrule

{Hardware-Hiding Module} & \mref{mHardware} \\
\midrule

\multirow{7}{0.3\textwidth}{Behaviour-Hiding Module}
& \mref{mCreateCar}\\
& \mref{mEvolveCar}\\
& \mref{mCreateRoad}\\
& \mref{mGraphicsDisplay}\\
& \mref{mGeneticAlgorithm}\\
\midrule

\multirow{3}{0.3\textwidth}{Software Decision Module} 
& \mref{mRandomSeed}\\
& \mref{mFitness}\\
& \mref{mSortingAndSearching}\\
& \mref{mPopulationGeneration}\\

\bottomrule

\end{tabular}
\caption{Module Hierarchy}
\label{TblMH}
\end{table}

\section{Connection Between Requirements and Design} \label{SecConnection}

The design of the system is intended to satisfy the requirements developed in
the SRS. In this stage, the system is decomposed into modules. The connection
between requirements and modules  and anticipated changes and modules is listed in Table \ref{TblRT}. 

\section{Module Decomposition} \label{SecMD}

Modules are decomposed according to the principle of ``information hiding''
proposed by \citet{ParnasEtAl1984}. The \emph{Secrets} field in a module
decomposition is a brief statement of the design decision hidden by the
module. The \emph{Services} field specifies \emph{what} the module will do
without documenting \emph{how} to do it. For each module, a suggestion for the
implementing software is given under the \emph{Implemented By} title. If the
entry is \emph{OS}, this means that the module is provided by the operating
system or by standard programming language libraries. 

Only the leaf modules in the
hierarchy have to be implemented. If a dash (\emph{--}) is shown, this means
that the module is not a leaf and will not have to be implemented. 

The system is designed to satisfy the requirements developed in the SRS. Throughout
this stage the system is decomposed into modules and analyzed.

\subsection{Hardware Hiding Modules (\mref{mHardware})}

\begin{description}
\item[Secrets:]The data structure and algorithm used to implement the virtual
  hardware.
\item[Services:]Serves as a virtual hardware used by the rest of the
  system. This module provides the interface between the hardware and the
  software. So, the system can use it to display outputs or to accept inputs.
\item[Implemented By:] OS
\end{description}

\subsection{Behaviour-Hiding Module}

\begin{description}
\item[Secrets:]The contents of the required behaviours.
\item[Services:]Includes programs that provide externally visible behaviour of
  the system as specified in the software requirements specification (SRS)
  documents. This module serves as a communication layer between the
  hardware-hiding module and the software decision module. The programs in this
  module will need to change if there are changes in the SRS.
\item[Implemented By:] --
\end{description}

\subsubsection{Car creation module}

\begin{description}
\item[Secrets:] Accepts a chromosome created by either the Genetic Algorithm module or the Random seed generation and manipulation module.
\item[Services:] Creates a car object according to the values present in the input chromosome (definition, x and y vertex arrays, wheel positions array, wheel radius array, and fitness).
\item[Implemented By:] Genetic Cars
\end{description}

\subsubsection{Evolve car module}

\begin{description}
\item[Secrets:] Takes population size number of cars as input.
\item[Services:] Creates the next generation of cars determined by the initial input generation by crossbreeding to create offspring as well as mutating offspring according to to a mutation rate.
\item[Implemented By:] Genetic Cars
\end{description}

\subsubsection{Road creation module}

\begin{description}
\item[Secrets:] Takes as input values created by the Random seed generation and manipulation module.
\item[Services:] Generates a randomly created road the becomes progressively steeper on which to simulate the cars.
\item[Implemented By:] Genetic Cars
\end{description}

\subsubsection{Graphics Display module}

\begin{description}
\item[Secrets:] A generation of cars and a complete road. Box 2d physics library.
\item[Services:] Creates a graphical representation of the simulation for the user to see as well as implementing the physics from a library of said simulation.
\item[Implemented By:] Genetic Cars
\end{description}

\subsection{Software Decision Module}

\begin{description}
\item[Secrets:] The design decision based on mathematical theorems, physical
  facts, or programming considerations. The secrets of this module are
  \emph{not} described in the SRS.
\item[Services:] Includes data structure and algorithms used in the system that
  do not provide direct interaction with the user. 
\item[Implemented By:] --
\end{description}

\subsubsection{Genetic Algorithm module}

\begin{description}
\item[Secrets:] The algorithms determining how each car generation changes from the last.
\item[Services:] Determines which parent cars creat the offspring. Creates the offspring. Mutates the genes in the offspring's chromosomes. Creates final chromosome to be used with the Create car and Evolve car modules.
\item[Implemented By:] Genetic Cars
\end{description}

\subsubsection{Random seed generation and manipulation module}

\begin{description}
\item[Secrets:] The algorithms determining the random elements of the application.
\item[Services:] Generates a random seed to be used for by the Car creation and Road creation modules. Generates random ints and floats for all of the behaviour hiding modules that use them.
\item[Implemented By:] Genetic Cars
\end{description}

\subsubsection{Fitness determination module}

\begin{description}
\item[Secrets:] The algorithms determining the fitness of  a car depending on its performance.
\item[Services:] Sets the criteria for score. Measures each car's score to determine which is the highest. Generates final fitness to be used by the Evolve car module.
\item[Implemented By:] Genetic Cars
\end{description}

\subsubsection{Sorting and Searching algorithms module}

\begin{description}
\item[Secrets:] The algorithms that search and sort through data (mostly in arrays)
\item[Services:] Brute force search and sort methods through arrays for the Create car, Evolve car, and Road creation modules. Uses selection sort for this purpose
\item[Implemented By:] Genetic Cars
\end{description}

\subsubsection{Population generation algorithms module}

\begin{description}
\item[Secrets:] The algorithms that effect the generation of the initial population and the proceding populations.
\item[Services:] Generates the initial population with all parameters. Generates the proceding populations using chromosomes generated in other modules.
\item[Implemented By:] Genetic Cars
\end{description}

\section{Traceability Matrix} \label{SecTM}

This section shows two traceability matrices: between the modules and the
requirements and between the modules and the anticipated changes. Requirements are outlined in greater detail in the SRS found \href{https://gitlab.cas.mcmaster.ca/linkk4/GrateBox/tree/master/Doc/SRS}{here}.

\begin{table}[H]
\centering
\begin{tabular}{p{0.2\textwidth} p{0.6\textwidth}}
\toprule
\textbf{Req.} & \textbf{Modules}\\
\midrule
Req 1: Car body parameters & \mref{mCreateCar}\\
Req 2: Wheel number parameters & \mref{mCreateCar}\\
Req 3: Wheel radius parameters & \mref{mCreateCar}\\
Req 4: Wheel position parameters & \mref{mCreateCar}\\
Req 5: Min weight parameters & \mref{mCreateCar}\\
Req 6: Max weight parameters & \mref{mCreateCar}\\
Req 7: Generation display parameters & \mref{mPopulationGeneration}, \mref{mGraphicsDisplay}\\
Req 8: Fitness display parameters & \mref{mFitness}, \mref{mGraphicsDisplay}\\
Req 9: Random seed parameters & \mref{mRandomSeed}\\
Req 10: Mutation rate parameters & \mref{mGeneticAlgorithm}\\
Req 11: Cars per generation parameters &\mref{mPopulationGeneration}\\ 
Req 12: Road generation parameters & \mref{mCreateRoad}\\
Req 13: Min cars per generation parameters & \mref{mPopulationGeneration}\\
Req 14: Max cars per generation parameters & \mref{mPopulationGeneration}\\
Req 15: Top cars parameters & \mref{mFitness}, \mref{mEvolveCar}, \mref{mGraphicsDisplay}\\
Req 16: Max top cars parameters & \mref{mFitness}, \mref{mSortingAndSearching}\\
Req 17: Min top cars parameters & \mref{mFitness}, \mref{mSortingAndSearching}\\
Req 18: Non-moving parameters & \mref{mCreateCar}, \mref{mGraphicsDisplay}\\
Req 19: Fitness paramaters & \mref{mFitness}, \mref{mSortingAndSearching}\\
Req 20: Default value replacement parameters & \mref{mCreateCar}, \mref{mCreateRoad}\\

\bottomrule
\end{tabular}
\caption{Trace Between Requirements and Modules}
\label{TblRT}
\end{table}

\begin{table}[H]
\centering
\begin{tabular}{p{0.2\textwidth} p{0.6\textwidth}}
\toprule
\textbf{AC} & \textbf{Modules}\\
\midrule
\acref{acWebsite} & \mref{mGraphicsDisplay}\\
\acref{acUserInput} & \mref{mCreateCar}, \mref{mCreateRoad}, \mref{mGeneticAlgorithm}, \mref{mPopulationGeneration}\\
\acref{acRoad} & \mref{mCreateRoad}\\
\acref{acAesthetics} & \mref{mGraphicsDisplay}\\
\acref{acGeneticAlgorithm} & \mref{mEvolveCar}, \mref{mGeneticAlgorithm}\\
\acref{acInitialFactors} & \mref{mCreateCar}, \mref{mPopulationGeneration}\\
\bottomrule
\end{tabular}
\caption{Trace Between Anticipated Changes and Modules}
\label{TblACT}
\end{table}

\section{Use Hierarchy Between Modules} \label{SecUse}

In this section, the uses hierarchy between modules is
provided. \citet{Parnas1978} said of two programs A and B that A {\em uses} B if
correct execution of B may be necessary for A to complete the task described in
its specification. That is, A {\em uses} B if there exist situations in which
the correct functioning of A depends upon the availability of a correct
implementation of B.  


\mref{mCreateCar} uses \mref{mRandomSeed} and \mref{mSortingAndSearching}.

\mref{mEvolveCar} uses \mref{mGeneticAlgorithm}, \mref{mFitness}, \mref{mSortingAndSearching}, and \mref{mRandomSeed}.

\mref{mCreateRoad} uses \mref{mRandomSeed}.

\mref{mGraphicsDisplay} uses \mref{mCreateCar} and \mref{mCreateRoad}.

\mref{mHardware}, \mref{mGeneticAlgorithm}, \mref{mRandomSeed}, \mref{mFitness}, \mref{mSortingAndSearching}, and \mref{mPopulationGeneration} are all independent modules.
%\section*{References}

\bibliographystyle {plainnat}
\bibliography {MG}

\end{document}