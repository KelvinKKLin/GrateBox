\documentclass[12pt, titlepage]{article}

\usepackage{booktabs}
\usepackage{tabularx}
\usepackage{hyperref}
\hypersetup{
    colorlinks,
    citecolor=black,
    filecolor=black,
    linkcolor=red,
    urlcolor=blue
}

\usepackage[round]{natbib}

\title{SE 3XA3: Requirements Document\\Genetic Cars}

\author{Team 8, Grate
		\\ Kelvin Lin (linkk4)
		\\ Eric Chaput (chaputem)
		\\ Jin Liu (liu456)
}

\date{\today}

%The follow 2 lines of code below were obtained from: 
%https://gitlab.cas.mcmaster.ca/ThisTooShallParse/3XA3_CParser/blob/master/Req_Update/L01_Group7_Requirements_Rev0.tex
\usepackage{mdframed}
\newmdenv[linecolor=black]{reqbox}


%% Comments

\usepackage{color}

\newif\ifcomments\commentstrue

\ifcomments
\newcommand{\authornote}[3]{\textcolor{#1}{[#3 ---#2]}}
\newcommand{\todo}[1]{\textcolor{red}{[TODO: #1]}}
\else
\newcommand{\authornote}[3]{}
\newcommand{\todo}[1]{}
\fi

\newcommand{\wss}[1]{\authornote{blue}{SS}{#1}}
\newcommand{\ds}[1]{\authornote{red}{DS}{#1}}
\newcommand{\mj}[1]{\authornote{red}{MSN}{#1}}
\newcommand{\cm}[1]{\authornote{red}{CM}{#1}}
\newcommand{\mh}[1]{\authornote{red}{MH}{#1}}

% team members should be added for each team, like the following
% all comments left by the TAs or the instructor should be addressed
% by a corresponding comment from the Team

\newcommand{\tm}[1]{\authornote{magenta}{Team}{#1}}


\begin{document}

\maketitle

\pagenumbering{roman}
\tableofcontents
\listoftables
\listoffigures

\begin{table}[bp]
\caption{\bf Revision History}
\begin{tabularx}{\textwidth}{p{3cm}p{2cm}X}
\toprule {\bf Date} & {\bf Version} & {\bf Notes}\\
\midrule
Date 1 & 1.0 & Notes\\
Date 2 & 1.1 & Notes\\
\bottomrule
\end{tabularx}
\end{table}

\newpage

\pagenumbering{arabic}

This document describes the requirements for ....  The template for the Software
Requirements Specification (SRS) is a subset of the Volere
template~\citep{RobertsonAndRobertson2012}.  If you make further modifications
to the template, you should explicity state what modifications were made.

\section{Project Drivers}

\subsection{The Purpose of the Project}

\subsection{The Stakeholders}

\subsubsection{The Client}

\subsubsection{The Customers}

\subsubsection{Other Stakeholders}

\subsection{Mandated Constraints}

\subsection{Naming Conventions and Terminology}

\subsection{Relevant Facts and Assumptions}

User characteristics should go under assumptions.

%KELVIN'S PART STARTS HERE
\section{Functional Requirements}

\subsection{The Scope of the Work and the Product}

\subsubsection{The Context of the Work}

\subsubsection{Work Partitioning}

\subsubsection{Individual Product Use Cases}

\newpage
\subsection{Functional Requirements}

%Box formatting template adapted from: 
%https://gitlab.cas.mcmaster.ca/ThisTooShallParse/3XA3_CParser/blob/master/Req_Update/L01_Group7_Requirements_Rev0.tex
\begin{reqbox}
%---
\begin{tabular}{ccc}
Requirement \#: & Requirement Type:  & Event/Use case \#: \\
\end{tabular} \\
%---
\textbf{Description:}   \\
\textbf{Rationale:}  \\
\textbf{Originator:} \\
\textbf{Fit Criterion:}  \\
%---
\begin{tabular}{ll}
\textbf{Customer Satisfaction:}  & \textbf{Customer Dissatisfaction:}  \\
\textbf{Priority:}  & \textbf{Conflicts:} \\
\end{tabular} \\
%---  
\textbf{Supporting Materials:}  \\
\textbf{History:} Created 
%--
\end{reqbox}


%KELVIN'S PART ENDS HERE

\section{Non-functional Requirements}

\subsection{Look and Feel Requirements}

As discussed in section 1.2 of this document, the users of this product include 
students and others interested in learning about genetic algorithms. With this 
in mind, the Genetic Cars project must be accessible to those without a 
background in mathematics or computer science. This accessibility begins with 
the look and feel of the project. The Genetic Cars project should appear 
aesthetically pleasing while still presenting its functions in as clean a manner 
as possible.

\subsubsection{Appearance Requirements}

The product shall be attractive to a student audience, with an emphasis on 
secondary and post-secondary students. A sampling of representative users shall, 
without prompting or enticement, be able to comprehend and use the product 
within sixty seconds of their first encounter with it. This same sampling shall 
also rate the appearance of the product on a scale from 1 to 10, and this rating 
shall be used to evaluate and refine the product's appearance. All licensing 
shall also be clear for the user to observe upon use of the product.

\subsubsection{Style Requirements}

The product shall appear inviting and educational and professional. After their 
first encounter with the product, a majority of representative users shall, 
without enticement, agree that they feel they would want to utilize the product 
and that they would learn about Genetic Algorithms by using the product. 
Representative users should also feel that they can trust the product.

\subsection{Usability and Humanity Requirements}

\subsubsection{Ease of Use Requirements}

The product shall be easy for anybody over the age of 6 to use. The product 
shall not expect the user to remember anything about the product given multiple 
uses. The product shall make the user want to use it and to show the product to 
their friends/family/etc.. The product shall be used by people with no training 
or education except for a basic knowledge of the English language and the most 
very basic functions of a computer, such as how to navigate to a web-site and 
how to enter inputs when prompted to do so. A representative sample of users 
shall be able to successfully complete a given set of tasks with the product 
within a specified period of time to be determined at the time of the sample. 
The representative sample shall also show a willingness to show the product to 
others.

\subsubsection{Personalization Requirements}

The product shall allow the user to make simple adjustments to the product to 
allow for a variable length and amount of trials depending on user input. 

\subsubsection{Learning Requirements}

The product shall be easy for an intended user of the product to learn. The 
product shall be able to be used by these users with no training before use. A 
representative sample of users shall be able to successfully complete a given 
set of tasks with the product within a specified period of time to be determined 
at the time of the sample.

\subsection{Performance Requirements}

\subsubsection{Speed and Latency Requirements}

The response time of the product shall be fast enough to avoid a loss of 
interest by the user following an input, which shall be a period of time no 
longer then five seconds. The initialization of the product shall be no longer 
then one minute.

\subsubsection{Precision and Reliability Requirements}

The product shall always converge towards a more optimal car. The product shall 
achieve 99 percent uptime. The product display shall be accurate to two decimal 
places.

\subsubsection{Longevity Requirements}

The product shall be easy to update and upgrade following its initial public 
release. 

\subsection{Operational and Environmental Requirements}

\subsubsection{Productization Requirements}

\subsection{Maintainability and Support Requirements}

\subsubsection{Maintenance Requirements}

\subsubsection{Supportability Requirements}

\subsubsection{Adaptability Requirements}

\subsection{Security Requirements}

\subsubsection{Access Requirements}

\subsubsection{Integrity Requirements}

\subsubsection{Privacy Requirements}

\subsection{Cultural Requirements}

\subsection{Legal Requirements}

\subsection{Health and Safety Requirements}

This section is not in the original Volere template, but health and safety are
issues that should be considered for every engineering project.

\section{Project Issues}

\subsection{Open Issues}

\subsection{Off-the-Shelf Solutions}

\subsection{New Problems}

\subsection{Tasks}

\subsection{Migration to the New Product}

\subsection{Risks}

\subsection{Costs}

\subsection{User Documentation and Training}

\subsection{Waiting Room}

\subsection{Ideas for Solutions}

\bibliographystyle{plainnat}

\bibliography{SRS}

\newpage

\section{Appendix}

%%This section has been added to the Volere template.  This is where you can 
place
%%additional information.

\subsection{List of Figures}

\subsection{Symbolic Parameters}

%Code Generated From http://www.tablesgenerator.com/#
\begin{table}[h!]
\centering
\label{LOF}
\begin{tabular}{ll}
Symbol & Definition \\
SOMERANDOMCONSTANT & SOMERANDOMDEFINITION  \\
ONEMORETIME & YES\\ 
\end{tabular}
\caption{List of Figures}
\end{table}


\end{document}
