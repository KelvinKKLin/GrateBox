\documentclass[12pt, titlepage]{article}

\usepackage{booktabs}
\usepackage{tabularx}
\usepackage{hyperref}
\hypersetup{
    colorlinks,
    citecolor=black,
    filecolor=black,
    linkcolor=red,
    urlcolor=blue
}

\usepackage[round]{natbib}

\title{SE 3XA3: Requirements Document\\Genetic Cars}

\author{Team 8, Grate
		\\ Kelvin Lin (linkk4)
		\\ Eric Chaput (chaputem)
		\\ Jin Liu (liu456)
}

\date{\today}

%The follow 2 lines of code below were obtained from: 
%https://gitlab.cas.mcmaster.ca/ThisTooShallParse/3XA3_CParser/blob/master/Req_Update/L01_Group7_Requirements_Rev0.tex
\usepackage{mdframed}
\newmdenv[linecolor=black]{reqbox}


%% Comments

\usepackage{color}

\newif\ifcomments\commentstrue

\ifcomments
\newcommand{\authornote}[3]{\textcolor{#1}{[#3 ---#2]}}
\newcommand{\todo}[1]{\textcolor{red}{[TODO: #1]}}
\else
\newcommand{\authornote}[3]{}
\newcommand{\todo}[1]{}
\fi

\newcommand{\wss}[1]{\authornote{blue}{SS}{#1}}
\newcommand{\ds}[1]{\authornote{red}{DS}{#1}}
\newcommand{\mj}[1]{\authornote{red}{MSN}{#1}}
\newcommand{\cm}[1]{\authornote{red}{CM}{#1}}
\newcommand{\mh}[1]{\authornote{red}{MH}{#1}}

% team members should be added for each team, like the following
% all comments left by the TAs or the instructor should be addressed
% by a corresponding comment from the Team

\newcommand{\tm}[1]{\authornote{magenta}{Team}{#1}}


\begin{document}

\maketitle

\pagenumbering{roman}
\tableofcontents
\listoftables
\listoffigures

\begin{table}[bp]
\caption{\bf Revision History}
\begin{tabularx}{\textwidth}{p{3cm}p{2cm}X}
\toprule {\bf Date} & {\bf Version} & {\bf Notes}\\
\midrule
October 7, 2016 & 1.0 & Started Functional Requirements\\
October 10, 2016 & 1.1 & Updated Functional Requirements\\
\bottomrule
\end{tabularx}
\end{table}

\newpage

\pagenumbering{arabic}

This document describes the requirements for ....  The template for the Software
Requirements Specification (SRS) is a subset of the Volere
template~\citep{RobertsonAndRobertson2012}.  If you make further modifications
to the template, you should explicity state what modifications were made.

\section{Project Drivers}

\subsection{The Purpose of the Project}

\subsection{The Stakeholders}

\subsubsection{The Client}

\subsubsection{The Customers}

\subsubsection{Other Stakeholders}

\subsection{Mandated Constraints}

\subsection{Naming Conventions and Terminology}

\subsection{Relevant Facts and Assumptions}

User characteristics should go under assumptions.

%KELVIN'S PART STARTS HERE
\section{Functional Requirements}

\subsection{The Scope of the Work and the Product}

\subsubsection{The Context of the Work}

\subsubsection{Work Partitioning}

\subsubsection{Individual Product Use Cases}

\newpage
\subsection{Functional Requirements}

%Box formatting template adapted from: 
%https://gitlab.cas.mcmaster.ca/ThisTooShallParse/3XA3_CParser/blob/master/Req_Update/L01_Group7_Requirements_Rev0.tex

%REQUIREMENT #1
\begin{reqbox}
%
\begin{tabular}{cc}
Requirement \#: 1 & Requirement Type: Functional \\
\end{tabular} \\
%
\textbf{Description:} The product must generate at least \textit{s} samples per generation. \\
\textbf{Rationale:}  GAs improve by having a large number of samples (representing members in a population) intermix traits. This requirement allows the GA to work by guaranteeing that a sufficient sample will be present at all times.\\
\textbf{Originator:} Kelvin Lin\\
\textbf{Fit Criterion:} Given a user generated input, \textit{s}, the program should generate \textit{s} cars for each generation.\\
%  
\textbf{Supporting Materials:} JavaScript \\
\textbf{History:} Created October 7\textsuperscript{th}, 2016
%
\end{reqbox}

%REQUIREMENT #2
\begin{reqbox}
%
\begin{tabular}{cc}
Requirement \#: 2 & Requirement Type: Functional \\
\end{tabular} \\
%
\textbf{Description:} Each car must be composed of at least \textit{v} vectors. \\
\textbf{Rationale:}  This requirement manages the complexity of the car model, allowing for realistic distribution of traits among members of a population. That is, this prevents large cars from being generated and using an excessive amount of memory. \\
\textbf{Originator:} Kelvin Lin\\
\textbf{Fit Criterion:} No car generated within population \textit{p} shall be composed of more than \textit{v} vectors.\\
%  
\textbf{Supporting Materials:} JavaScript \\
\textbf{History:} Created October 7\textsuperscript{th}, 2016
%
\end{reqbox}

\newpage

%REQUIREMENT #3
\begin{reqbox}
%
\begin{tabular}{cc}
Requirement \#: 3 & Requirement Type: Functional \\
\end{tabular} \\
%
\textbf{Description:} Each car may not have more than \textit{number\_of\_vertices} wheels. \\
\textbf{Rationale:}  The wheels must be attached to the car via a vertex between two connecting vectors. This requirement ensures that no redundant or unused wheels will be generated.\\
\textbf{Originator:} Kelvin Lin\\
\textbf{Fit Criterion:} No car generated within population \textit{p} shall be composed of more than \textit{number\_of\_vertices} wheels.\\
%  
\textbf{Supporting Materials:} JavaScript \\
\textbf{History:} Created October 7\textsuperscript{th}, 2016
%
\end{reqbox}

%REQUIREMENT #4
\begin{reqbox}
%
\begin{tabular}{cc}
Requirement \#: 4 & Requirement Type: Functional \\
\end{tabular} \\
%
\textbf{Description:} The center of each wheel generated must be attached to a vertex formed by connecting vectors. \\
\textbf{Rationale:}  Wheels cannot be floating on or around the car. This requirement ensures visual coherency by requiring wheels to be attached to the car model. Knowing the center of the wheel will also allow the physics engine to calculate the torque and distance that the car travelled.\\
\textbf{Originator:} Kelvin Lin\\
\textbf{Fit Criterion:} Each wheel displayed on the screen is attached to a vertex formed by connecting vectors.\\
%  
\textbf{Supporting Materials:} JavaScript \\
\textbf{History:} Created October 7\textsuperscript{th}, 2016
%
\end{reqbox}

\newpage

%REQUIREMENT #5
\begin{reqbox}
%
\begin{tabular}{cc}
Requirement \#: 5 & Requirement Type: Functional \\
\end{tabular} \\
%
\textbf{Description:} The radius of each wheel must be at most \textit{r} units. \\
\textbf{Rationale:}  This requirement manages the complexity of the car model, allowing for realistic distribution of traits among members of a population. That is, cars with unrealistically sized wheels will not be generated.\\
\textbf{Originator:} Kelvin Lin\\
\textbf{Fit Criterion:} No cars generated will have wheels with a radius larger than \textit{r}.\\
%  
\textbf{Supporting Materials:} JavaScript \\
\textbf{History:} Created October 7\textsuperscript{th}, 2016
%
\end{reqbox}

%REQUIREMENT #6
\begin{reqbox}
%
\begin{tabular}{cc}
Requirement \#: 6 & Requirement Type: Functional \\
\end{tabular} \\
%
\textbf{Description:}  \\
\textbf{Rationale:}  \\
\textbf{Originator:} Kelvin Lin\\
\textbf{Fit Criterion:} \\
%  
\textbf{Supporting Materials:} JavaScript \\
\textbf{History:} Created October 10\textsuperscript{th}, 2016
%
\end{reqbox}

\newpage

%REQUIREMENT #7
\begin{reqbox}
%
\begin{tabular}{cc}
Requirement \#: 7 & Requirement Type: Functional \\
\end{tabular} \\
%
\textbf{Description:}  \\
\textbf{Rationale:}  \\
\textbf{Originator:} Kelvin Lin\\
\textbf{Fit Criterion:} \\
%  
\textbf{Supporting Materials:} JavaScript \\
\textbf{History:} Created October 10\textsuperscript{th}, 2016
%
\end{reqbox}

%REQUIREMENT #8
\begin{reqbox}
%
\begin{tabular}{cc}
Requirement \#: 8 & Requirement Type: Functional \\
\end{tabular} \\
%
\textbf{Description:}  \\
\textbf{Rationale:}  \\
\textbf{Originator:} Kelvin Lin\\
\textbf{Fit Criterion:} \\
%  
\textbf{Supporting Materials:} JavaScript \\
\textbf{History:} Created October 10\textsuperscript{th}, 2016
%
\end{reqbox}

\newpage

%REQUIREMENT #9
\begin{reqbox}
%
\begin{tabular}{cc}
Requirement \#: 9 & Requirement Type: Functional \\
\end{tabular} \\
%
\textbf{Description:}  \\
\textbf{Rationale:}  \\
\textbf{Originator:} Kelvin Lin\\
\textbf{Fit Criterion:} \\
%  
\textbf{Supporting Materials:} JavaScript \\
\textbf{History:} Created October 10\textsuperscript{th}, 2016
%
\end{reqbox}

%REQUIREMENT #10
\begin{reqbox}
%
\begin{tabular}{cc}
Requirement \#: 10 & Requirement Type: Functional \\
\end{tabular} \\
%
\textbf{Description:}  \\
\textbf{Rationale:}  \\
\textbf{Originator:} Kelvin Lin\\
\textbf{Fit Criterion:} \\
%  
\textbf{Supporting Materials:} JavaScript \\
\textbf{History:} Created October 10\textsuperscript{th}, 2016
%
\end{reqbox}

\newpage

%KELVIN'S PART ENDS HERE

\section{Non-functional Requirements}

\subsection{Look and Feel Requirements}

As discussed in section 1.2 of this document, the users of this product include 
students and others interested in learning about genetic algorithms. With this 
in mind, the Genetic Cars project must be accessible to those without a 
background in mathematics or computer science. This accessibility begins with 
the look and feel of the project. The Genetic Cars project should appear 
aesthetically pleasing while still presenting its functions in as clean a manner 
as possible.

\subsubsection{Appearance Requirements}

The product shall be attractive to a student audience, with an emphasis on 
secondary and post-secondary students. A sampling of representative users shall, 
without prompting or enticement, be able to comprehend and use the product 
within sixty seconds of their first encounter with it. This same sampling shall 
also rate the appearance of the product on a scale from 1 to 10, and this rating 
shall be used to evaluate and refine the product's appearance. All licensing 
shall also be clear for the user to observe upon use of the product.

\subsubsection{Style Requirements}

The product shall appear inviting and educational and professional. After their 
first encounter with the product, a majority of representative users shall, 
without enticement, agree that they feel they would want to utilize the product 
and that they would learn about Genetic Algorithms by using the product. 
Representative users should also feel that they can trust the product.

\subsection{Usability and Humanity Requirements}

\subsubsection{Ease of Use Requirements}

The product shall be easy for anybody over the age of 6 to use. The product 
shall not expect the user to remember anything about the product given multiple 
uses. The product shall make the user want to use it and to show the product to 
their friends/family/etc.. The product shall be used by people with no training 
or education except for a basic knowledge of the English language and the most 
very basic functions of a computer, such as how to navigate to a web-site and 
how to enter inputs when prompted to do so. A representative sample of users 
shall be able to successfully complete a given set of tasks with the product 
within a specified period of time to be determined at the time of the sample. 
The representative sample shall also show a willingness to show the product to 
others.

\subsubsection{Personalization Requirements}

The product shall allow the user to make simple adjustments to the product to 
allow for a variable length and amount of trials depending on user input. 

\subsubsection{Learning Requirements}

The product shall be easy for an intended user of the product to learn. The 
product shall be able to be used by these users with no training before use. A 
representative sample of users shall be able to successfully complete a given 
set of tasks with the product within a specified period of time to be determined 
at the time of the sample.

\subsection{Performance Requirements}

\subsubsection{Speed and Latency Requirements}

The response time of the product shall be fast enough to avoid a loss of 
interest by the user following an input, which shall be a period of time no 
longer then five seconds. The initialization of the product shall be no longer 
then one minute.

\subsubsection{Precision and Reliability Requirements}

The product shall always converge towards a more optimal car. The product shall 
achieve 99 percent uptime. The product display shall be accurate to two decimal 
places.

\subsubsection{Longevity Requirements}

The product shall be easy to update and upgrade following its initial public 
release. 

\subsection{Operational and Environmental Requirements}

\subsubsection{Productization Requirements}

\subsection{Maintainability and Support Requirements}

\subsubsection{Maintenance Requirements}

\subsubsection{Supportability Requirements}

\subsubsection{Adaptability Requirements}

\subsection{Security Requirements}

\subsubsection{Access Requirements}

\subsubsection{Integrity Requirements}

\subsubsection{Privacy Requirements}

\subsection{Cultural Requirements}

\subsection{Legal Requirements}

\subsection{Health and Safety Requirements}

This section is not in the original Volere template, but health and safety are
issues that should be considered for every engineering project.

\section{Project Issues}

\subsection{Open Issues}
Not applicable for this project.

\subsection{Off-the-Shelf Solutions}
Not applicable for this project.

\subsection{New Problems}
There is a risk that the copyright holder of Box Car 2D does not let anyone
 else use their codes anymore. In addition, if any developer in our group 
leave the group or drop the class in the future, this project will be 
difficult to implement since every developer is doing his own part and
 information will be gaped.

\subsection{Tasks}
Car modeling, Genetic algorithm design and graphics design will be doing 
concurrently and tested thoroughly. User interface will be designed after 
graphics done and project will be hosted on GitLab after codes implemented.

\subsection{Migration to the New Product}
Not applicable for this project.

\subsection{Risks}
The Box2D API poses the most significant risk for the car model. The Box2D 
API defines the car entity in terms that can be used with many physics 
equations, which is important for calculating the fitness function of the 
car. In the event that the Box2D API proves to be infeasible for Team 8, 
alternate arrangements will have to be made in order to complete the
project: the team will resort to using basic kinematics equations to calculate 
the fitness function instead of using the API. A possible drawback to this 
approach would be that the members of Team 8 are generally unfamiliar with 
Newtonian mechanics, so external assistance would be required. 

\subsection{Costs}
There will be no cost at all since all the software (Latex editor, code 
complier etc.) and web-hosting are free. 

\subsection{User Documentation and Training}
The user documents will be simple and efficient for our project since this 
project will not ask the user to do many things. The main responsibility for 
training documentation is letting user familiar with the start button, reset 
button and output table.

\subsection{Waiting Room}
Audio effect is expected to be add to this project.

\subsection{Ideas for Solutions}
Good structure and design for this project.


\bibliographystyle{plainnat}

\bibliography{SRS}

\newpage

\section{Appendix}

%%This section has been added to the Volere template.  This is where you can 
place
%%additional information.

\subsection{List of Figures}

\subsection{Symbolic Parameters}

%Code Generated From http://www.tablesgenerator.com/#
\begin{table}[h!]
\centering
\label{LOF}
\begin{tabular}{ll}
Symbol & Definition \\
\textit{s} & The number of samples in a generation  \\
\textit{v} & The number of vectors in a car  \\
\textit{number\_of\_vertices} & The number of vertices formed by connecting vectors in a car model \\
\textit{r} & The radius of a wheel
\end{tabular}
\caption{List of Figures}
\end{table}


\end{document}
