\documentclass[12pt, titlepage]{article}

\usepackage{fullpage}
\usepackage[round]{natbib}
\usepackage{multirow}
\usepackage{booktabs}
\usepackage{tabularx}
\usepackage{graphicx}
\usepackage{float}
\usepackage{hyperref}
\hypersetup{
    colorlinks,
    citecolor=black,
    filecolor=black,
    linkcolor=red,
    urlcolor=blue
}
\usepackage[round]{natbib}

\newcounter{mnum}
\newcommand{\mthemnum}{M\themnum}
\newcommand{\mref}[1]{M\ref{#1}}

\title{SE 3XA3: Test Report\\GrateBox}

\author{Team 8, Grate
		\\ Kelvin Lin (linkk4)
		\\ Eric Chaput (chaputem)
		\\ Jin Liu (liu456)
}

\date{\today}

%% Comments

\usepackage{color}

\newif\ifcomments\commentstrue

\ifcomments
\newcommand{\authornote}[3]{\textcolor{#1}{[#3 ---#2]}}
\newcommand{\todo}[1]{\textcolor{red}{[TODO: #1]}}
\else
\newcommand{\authornote}[3]{}
\newcommand{\todo}[1]{}
\fi

\newcommand{\wss}[1]{\authornote{blue}{SS}{#1}}
\newcommand{\ds}[1]{\authornote{red}{DS}{#1}}
\newcommand{\mj}[1]{\authornote{red}{MSN}{#1}}
\newcommand{\cm}[1]{\authornote{red}{CM}{#1}}
\newcommand{\mh}[1]{\authornote{red}{MH}{#1}}

% team members should be added for each team, like the following
% all comments left by the TAs or the instructor should be addressed
% by a corresponding comment from the Team

\newcommand{\tm}[1]{\authornote{magenta}{Team}{#1}}


\begin{document}

\maketitle

\pagenumbering{roman}
\tableofcontents
\listoftables
\listoffigures

\begin{table}[bp]
\caption{\bf Revision History}
\begin{tabularx}{\textwidth}{p{3cm}p{2cm}X}
\toprule {\bf Date} & {\bf Version} & {\bf Notes}\\
\midrule
Dec 6 & 1.0 & Document creation\\
Dec 7 & 1.1 & Document completion\\
\bottomrule
\end{tabularx}
\end{table}

\newpage

\pagenumbering{arabic}

This document will make frequent reference to the Development, Test Plan, SRS 
and Design documents of the GrateBox project. They can be found 
\href{https://gitlab.cas.mcmaster.ca/linkk4/GrateBox/tree/master/Doc/DevelopmentPlan}{here}, 
\href{https://gitlab.cas.mcmaster.ca/linkk4/GrateBox/tree/master/Doc/TestPlan}{here}, 
\href{https://gitlab.cas.mcmaster.ca/linkk4/GrateBox/tree/master/Doc/SRS}{here}, 
and 
\href{https://gitlab.cas.mcmaster.ca/linkk4/GrateBox/tree/master/Doc/Design}{here} 
respectively.

\section{Functional Requirements Evaluation}

Functional requirements were evaluated through automated and non-automated 
testing.

\subsection{Automated Testing}

Automated testing for GrateBox was conducted using the QUnit software outlined 
in the Development document. The test cases executed can be found in the test.js 
file 
\href{https://gitlab.cas.mcmaster.ca/linkk4/GrateBox/tree/master/src/test}{here}. 
Each test case in the test.js file corresponds with an automated test case in 
the Test Plan document. The automated test cases in the Test Plan document are 
those found in sections 3.1.1 and 3.1.2. All automated test cases were properly 
implemented and executed, and the results returned were the results expected, 
indicating successful implementation of functional requirements by automated 
testing.

\subsection{Non-Automated Testing}

Non-Automated testing for GrateBox's functional requirements was conducted by 
the test team outlined in the Test Plan document. These include the tests 
outlined in sections 3.1.3, 3.1.4, and 3.1.5 of the Test Plan document. All test 
cases were executed correctly and returned the results expected. They are broken 
down into more detail below.

\subsubsection{Graphics}

Test for graphics are as follows.\\

GR-1.1\\

Cars are generated as expected given numerical values. The position of vertices, 
the connections between vertices, the radius of wheels, and the placement of 
wheels all vary depending on given input. Comparison with BoxCar-2D also 
verifies successful implementation of graphics module. Test successful.\\

GR-1.2\\

Cars are not generated as expected given invalid numerical values. An error 
message is displayed when this occurs. Test successful.\\

GR-1.3\\

Cars are not generated as expected given invalid numerical values. An error 
message is displayed when this occurs. Test successful.\\

GR-2.1\\

Road created corresponds to algorithm input. Test successful.\\

GR-2.2\\

No road created and error message is displayed when this occurs. Test 
successful.

\subsubsection{Fitness and Score}

Test for fitness and scores are as follows.\\

FI-1\\

Values calculated correspond to values observed. Test successful.\\

FI-2\\

Values calculated correspond to values observed and value properly displayed in 
GUI. Test successful.

\subsubsection{Other GUI elements}

Test for other GUI elements are as follows.\\

GU-1\\

Health bars operate properly. Test successful.\\

GU-2\\

Text file is created and is accurate. Test successful.

\section{Nonfunctional Requirements Evaluation}

\subsection{Nonfunctional Tests}

The exact details of non-functional requirements can be found in the Test Plan 
document in section 3.2.


\subsubsection{Look and Feel}

LF-1\\

Majority of users agreed that the visual aesthetic of the program rated 
favourable. Test successful.\\

LF-2\\

Majority of users agreed that the style of the program rated favourable. Test 
successful.

\subsubsection{Usability}

US-1\\

Users performed all tasks in allotted time. Test successful.\\

US-2\\

Users performed all tasks in allotted time. Test successful.\\

US-3\\

Majority of users agreed that the program's usability rated favourable. Test 
successful.

\subsubsection{Performance}

PF-1\\

Time restriction for tasks performed met. Test successful.\\

PF-2\\

Majority of users agreed that the program's usability rated favourable. Test 
successful.\\

PF-3\\

Numerical values and equations determined to be accurate and valid. Test 
successful.\\

PF-4\\

Majority of users agreed that the program's usability rated favourable. Test 
successful.

\subsection{Overall Evaluation valuation based on tests and user survey}

Overall evaluation of non-functional requirements takes into account the results 
of the final user surveys and the results of the above test cases. The nebulous 
nature of non-functional requirements means that exact evaluation is difficult. 
With this in mind, the GrateBox project achieved most of the goals set out by 
non-functional requirements. GrateBox looks and feels good, is highly usable, 
and performs as expected. All fit criterion were met or exceeded and users 
agreed that our product excelled non-functionally. Stress testing, while not 
directly applicable to GrateBox (GrateBox is a simple program executed by one 
machine by one user), has been undertaken indirectly through a 2 hour execution 
of GrateBox on two different machines, demonstrating the robustness of the 
program.

\section{Comparison to Existing Implementation}

The original implementation contained no testing of its own. One test conducted 
by Grate (GR-1.1) required a comparison to the original implementation. This 
comparison helped validate the results of the test as seen in section 1.2.1.

\section{Unit Testing}

All unit testing for this project can be found in the test folder found 
\href{https://gitlab.cas.mcmaster.ca/linkk4/GrateBox/tree/master/src/test}{here}. 
The tests.java file contains the source code for all testing. The Test.html file 
contains the output of this code. All test were conducted using the third party 
testing software QUnit, outlined in the Design document.

\section{Changes Due to Testing}

Several changes were made to GrateBox as a result of testing. They have been 
divided below according to their related requirements.

\subsection{Changes to functional requirements}

Initial car design was predicated on the creation of vectors from a centre point 
and then the connecting of the end of these vectors to form a polygon that would 
serve as the body of the car. However, testing for car creation using this 
method proved prohibitively difficult. To this end, car bodies were redesigned 
to be composed of a series of conducted vertexes. This change also affected how 
car chromosomes were created and manipulated.\\


User surveys revealed the need for a pause button in GrateBox as users felt a 
lack of control over the program at times. This was added to the parameters 
section of GrateBox.\\


The initial values of the user variables (number of cars per generation, number 
of parents, and mutation rate) all increased as a result of user input. The 
initial values resulted in cars that were too slow in evolving, and would often 
require ten or more generations for a moderately successful car to emerge. 

\subsection{Changes to non-functional requirements}

User feedback told Grate that the UI of GrateBox was too minimal, and that look 
and feel requirements were not being met. To this end, an html visual enhancer, 
BootStrap, was used to satisfy this user need. Users opinions' of GrateBox's 
look and feel increased dramatically following this change.\\


It was observed during user trials that many users had trouble grasping the 
exact nature of genetic algorithms quickly. To this end text was added to 
GrateBox to give users the most very basic background with which to use the 
program. It was important to Grate that this text not be overly extensive, as it 
was believed that this could go a long way to turning users off of our product. 
The exact amount and content of text was modified extensively over GrateBox's 
development cycle via user input.

\section{Automated Testing}

Automated testing proved difficult for certain elements of GrateBox, as it is 
by nature a visual product. Still some automated testing was conducted for the 
benefit of the testing team and to improve accuracy. QUnit, our unit testing 
software, allows for multiple executions of the same test with some variances, 
and this was done to ensure the soundness of many aspects of GrateBox. This 
functionality allows for the entrance of a variable into a field with the 
instructions to manipulate that variable within a range of possible values and 
determine many possible outputs (for example negative values, non-integer 
values, etc.). The genetic algorithm in particular benefited greatly from 
automated testing, as many different variables could be tested with minimal time 
investment. While it proved impractical for the project given time 
constrictions, further elaboration on GrateBox could utilise image analysing on 
a pixel by pixel basis to improve the quality of the graphical output. Although 
the overall value of such a time investment is questionable.
		
\section{Trace to Requirements}

The requirements are described in more detail in the SRS document. The tests are 
given by their abbreviated forms. For example GA refers to the Genetic Algorithm 
tests, which are described in more detail in the Test Plan document.
		
\begin{table}[H]
\centering
\begin{tabular}{p{0.2\textwidth} p{0.6\textwidth}}
\toprule
\textbf{Req.} & \textbf{Tests}\\
\midrule
Req 1: Car body parameters & CM \\
Req 2: Wheel number parameters & CM \\
Req 3: Wheel radius parameters & CM\\
Req 4: Wheel position parameters & CM\\
Req 5: Min weight parameters & CM\\
Req 6: Max weight parameters & CM\\
Req 7: Generation display parameters & GR\\
Req 8: Fitness display parameters & GA, FI\\
Req 9: Random seed parameters & GA\\
Req 10: Mutation rate parameters & GA\\
\bottomrule
\end{tabular}
\caption{Trace between functional requirements and tests (set 1)}
\label{TblRT}
\end{table}

\begin{table}[H]
\centering
\begin{tabular}{p{0.2\textwidth} p{0.6\textwidth}}
\toprule
\textbf{Req.} & \textbf{Tests}\\
\midrule
Req 11: Cars per generation parameters  & GA\\
Req 12: Road generation parameters & CM\\
Req 13: Min cars per generation parameters & CM\\
Req 14: Max cars per generation parameters & CM\\
Req 15: Top cars parameters & CM, FI\\
Req 16: Max top cars parameters & CM, GA, FI\\
Req 17: Min top cars parameters & CM, GA, FI\\
Req 18: Non-moving parameters & CM, GU\\
Req 19: Fitness parameters & FI\\
Req 20: Default value replacement parameters & GU\\
\bottomrule
\end{tabular}
\caption{Trace between functional requirements and tests (set 2)}
\label{TblRT2}
\end{table}

\begin{table}[H]
\centering
\begin{tabular}{p{0.2\textwidth} p{0.6\textwidth}}
\toprule
\textbf{Req.} & \textbf{Tests}\\
\midrule
Req 21: Appearance Parameters & LF\\
Req 22: Style Parameters & LF\\
Req 23: Ease of Use Parameters & US\\
Req 24: Personalization Parameters & LF\\
Req 25: Learning Parameters & US\\
Req 26: Speed and Latency Parameters & PF\\
Req 27: Precision and Reliability Parameters & PF\\
Req 28: Longevity Parameters & PF\\
\bottomrule
\end{tabular}
\caption{Trace between non-functional requirements and tests}
\label{TblRT2}
\end{table}

\section{Trace to Modules}	

For reference, the modules are as follows. There are displayed with more detail 
in the Design document. All tests are referred to in their abbreviated form. For 
example Genetic Algorithm tests are referred to by the short form GA. These are 
elaborated on in the Test Plan document.

\begin{description}
\item [\refstepcounter{mnum} \mthemnum \label{mHardware}:] Hardware hiding 
module
\item [\refstepcounter{mnum} \mthemnum \label{mCreateCar}:]  Car creation 
module. Tested with the CM tests.
\item [\refstepcounter{mnum} \mthemnum \label{mEvolveCar}:] Evolve car module. 
Tested with the CM tests.
\item [\refstepcounter{mnum} \mthemnum \label{mCreateRoad}:] Road creation 
module. Tested with the CM tests.
\item [\refstepcounter{mnum} \mthemnum \label{mGraphicsDisplay}:] Graphics 
display module. Tested with the GR and GU tests.
\item [\refstepcounter{mnum} \mthemnum \label{mGeneticAlgorithm}:] Genetic 
Algorithm module. Tested with the GA tests.
\item [\refstepcounter{mnum} \mthemnum \label{mRandomSeed}:] Random seed 
generation and manipulation module. Tested with the GA tests.
\item [\refstepcounter{mnum} \mthemnum \label{mFitness}:] Fitness determination 
module. Tested with the FI tests.
\item [\refstepcounter{mnum} \mthemnum \label{mSearching}:] Searching algorithms 
module. Tested with the GA tests.
\item [\refstepcounter{mnum} \mthemnum \label{mSorting}:] Sorting algorithms 
module. Tested with the GA tests.
\item [\refstepcounter{mnum} \mthemnum \label{mPopulationGeneration}:] 
Population generation algorithms module. Tested with the GA tests.
\end{description}


\section{Code Coverage Metrics}

In the Test Plan document, Grate endeavoured to achieve 70\% coverage with our 
automated and unit testing. While not every unit test possible was executed, 
most were, and Grate feels that this coverage metric was reached, as a large 
majority of our actual code contains corresponding tests to validate it. The 
modularized nature of our code reduced the overall usefulness of larger coverage 
tests. For example, the inclusion of the genetic algorithm in a test for 
graphical validity achieves very little, as the modularized nature of the code 
requires little to no interaction between these two modules.

\end{document}
