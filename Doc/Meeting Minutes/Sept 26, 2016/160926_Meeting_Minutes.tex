%%The following template was modified from the source code provided at: 
%%https://www.overleaf.com/latex/templates/a-simple-template-for-meeting-minutes
%%/qzszxjhvsnxz#.V-iBpPArJhE

%%
%% This is file `./samples/minutes.tex',
%% generated with the docstrip utility.
%%
%% The original source files were:
%%
%% meetingmins.dtx  (with options: `minutes')
%% ----------------------------------------------------------------------
%% 
%% meetingmins - A LaTeX class for formatting minutes of meetings
%% 
%% Copyright (C) 2011-2013 by Brian D. Beitzel <brian@beitzel.com>
%% 
%% This work may be distributed and/or modified under the
%% conditions of the LaTeX Project Public License (LPPL), either
%% version 1.3c of this license or (at your option) any later
%% version.  The latest version of this license is in the file:
%% 
%% http://www.latex-project.org/lppl.txt
%% 
%% Users may freely modify these files without permission, as long as the
%% copyright line and this statement are maintained intact.
%% 
%% ----------------------------------------------------------------------
%% 

\documentclass[11pt]{meetingmins}

\setcommittee{Team 8: Grate}

\setmembers{
  \chair{Kelvin Lin},
  Eric Chaput,
  Jin Liu
}

\setdate{September 26, 2016}

\setpresent{
  \chair{Kelvin Lin},
  Eric Chaput,
  Jin Liu
}


\begin{document}
\maketitle

\section{Introduction}
The purpose of this meeting is to formalize the higher level aspects of our group. The meeting will begin by discussing our formal expectations for each other, followed by our expectations of this project. The meeting will then scrutinize each slide in Professor Spencer Smith's Lecture 2 Presentation, and decide upon formalities proposed in those slides. The meeting will conclude with an official decision on the approach Team 8 will take to all deliverables up to the proof-of-concept demonstration. 

\section{Overview}
\begin{hiddenitems}

\item
What are our expectations for each other this term?

\item
What is our meeting plan for the semester?

\item
What is our team communication plan?

\item
What are the roles of each team member?

\item
How will Git be used?

\item
What technologies are we using?

\item
What coding conventions are we going to adhere to?

\item
What is our anticipated project schedule?

\item
What do we want to have for our Proof of Concept Demonstration?

\end{hiddenitems}

\section{Extended Overview}
\begin{hiddenitems}
\item
What are our expectations for each other this term?
\begin{itemize}
	\item{How much time should each person put into the group?}
	\item{How strictly do we expect each person to adhere to deadlines?}
	\item{What should a team member do if he cannot finish a task on time? How much advance
	      notice should be given?}
	\item{What are our expectations for each other regarding punctuality (i.e. lateness, absence)}
	\item{What are our expectations for quality of work? What is our process for approaching a
	      team member who is consistently producing low quality work?}
	\item{What are the consequences for not adhering to these expectations?}
\end{itemize}

\item
What is our meeting plan for the semester?
\begin{itemize}
	\item{How often are we going to meet?}
	\item{Where are we going to meet? If we are meeting in a room, who will be responsible for
	      booking the room?}
	\item{What time are we going to meet?}
	\item{Are meetings flexible? Can meetings be deferred or skipped?}
	\item{Can we have more meetings if needed? If so, what are the best times to have more
	      meetings?}
	\item{Who is in charge of making the agenda?}
	\item{Who is in charge of taking notes?}
	
\end{itemize}

\item
What is our team communication plan?
\begin{itemize}
	\item{What are our sources of communication?}
	\item{How is each source of communication to be used?}
	\item{What is the priority of our sources of communication?}
	\item{How often are we expected to check each source of communication?}
\end{itemize}

\item
What are the roles of each team member?
\begin{itemize}
	\item{Will there be a team leader?}
	\item{Will there be a scribe?}
	\item{Who has specific expertise in this group?}
\end{itemize}

\item
How will Git be used?
\begin{itemize}
	\item{How are we going to structure the repository?}
	\item{How much code should we write before pushing to Git?}
	\item{How will labels be used?}
	\item{How will milestones be used?}
\end{itemize}

\item
What technologies are we using?
\begin{itemize}
	\item{What languages are we using? What versions?}
	\item{What IDEs are we using?}
	\item{What testing frameworks are we using?}
	\item{What document generator are we using?}
\end{itemize}

\item
What coding conventions are we going to adhere to?

\item
What is our anticipated project schedule? (Refer to Appendix A)

\item
What do we want to have for our Proof of Concept Demonstration?
\begin{itemize}
	\item{What are the risks involved?}
	\item{How do we mitigate the risks?}
	\item{What is our contingency plan?}
\end{itemize}
\end{hiddenitems}

\section{Meeting Notes}
	\subsection{What are our expectations for each other this term?}
		We are expected to work 5 hours on a regular week, 10 hours on a busy week.If any of us don't preform, then we'll talk to him, and if he 			continue to not preform, we'll talk to him with the prof.If anyone does things late, let everyone else know so we can help.
		

		
	\subsection{What is our meeting plan for the semester?}
		1.Monday 7pm-8pm.  
		2.Wednesday and Friday during the lab.

		
	\subsection{What is our team communication plan?}
		1.Facebook for regular communication (check it every day).
		2.Phone for emergency.

		
	\subsection{What are the roles of each team member?}
		1.Team leader and technology expert: Kelvin
		2.Notes and graphics: Jin 
		3.Documentation: Eric
		
	\subsection{How will Git be used?}
		1.Using master branch for completed documents.
		2.Create a progress branch for in-progress documents.

		
	\subsection{What technologies are we using?}
		1.JavaScript vanilla.
		2.Canvas on HTML5.

	\subsection{What coding conventions are we going to adhere to?}
		Mozilla Developer Network .
		
	\subsection{What is our anticipated project schedule?}
		1.Development plan on Sep 28th.
		2.Requirement documents revision 0 on Oct 5th.
		3.Proof of concept on Oct 16th.
		4.Test plan revision 0 on Oct 26th.
		5.Design documents revision on Nov 9th.
		6.Revision 0 demonstration on Nov 14th.		
		7.Final demonstration on Nov 28th.
		8.Final documentation on Dec 6th.

		
	\subsection{What do we want to have for our Proof of Concept Demonstration?}
		1.Make sure one of the algorithms works.
		2.Apply the algorithm to the car.
		3.Coded.
		4.Make sure graphics work.
		5.Design user Interface.

		
\section{Next Meeting}
	Wednesday, September 28th at 8:00
	\vspace{1em}
	\nextmeeting{Wednesday, October 19, at 3:00}

\newpage
\section{Appendix A: Milestones and Deadlines}
\begin{tabular}{ p{6.7cm} l}

  Team Formation & Week of September 12\\

  Project Approval & Week of September 19\\

  Problem Statement & September 23\\

  Development Plan & September 30\\

  Requirements Document Revision 0 & October 7\\

  Proof of Concept Demonstration & Week of October 17\\

  Test Plan Revision 0 & October 28\\

  Design Document Revision 0 & November 11\\

  Revision 0 Demonstration & Week of November 14\\

  Lab Exercises & Throughout Term\\

  Final Demonstration (Revision 1) & Week of November 28\\

  Peer Evaluation of Other Team & Week of November 28\\

  Final Documentation (Revision 1) & December 8\\

\end{tabular}

\newpage
\section{Appendix B: Summary of Final Decisions}
	For the proof of concept, our main target is one of the genetic alogorithms and the visual graphics(like generate 1 polygun first). User interface is counted as well. And there are some risks we may face such as physics library or HTML5 Canvas doesn't work. As to the project timeline, all the parts should be done and checked 3 days before due date so we can make sure we dont mess up everything.




\end{document}

%% 
%% Copyright (C) 2011-2013 by Brian D. Beitzel <brian@beitzel.com>
%% 
%% This work may be distributed and/or modified under the
%% conditions of the LaTeX Project Public License (LPPL), either
%% version 1.3c of this license or (at your option) any later
%% version.  The latest version of this license is in the file:
%% 
%% http://www.latex-project.org/lppl.txt
%% 
%% Users may freely modify these files without permission, as long as the
%% copyright line and this statement are maintained intact.
%% 
%% This work is "maintained" (as per LPPL maintenance status) by
%% Brian D. Beitzel.
%% 
%% This work consists of the file  meetingmins.dtx
%% and the derived files           meetingmins.cls,
%%                                 sampleminutes.tex,
%%                                 department.min,
%%                                 README.txt, and
%%                                 meetingmins.pdf.
%% 
%%
%% End of file `./samples/minutes.tex'.