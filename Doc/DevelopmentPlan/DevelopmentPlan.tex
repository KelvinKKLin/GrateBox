\documentclass{article}

\usepackage{booktabs}
\usepackage{tabularx}

\title{SE 3XA3: Development Plan\\Genetic Cars}

\author{Team 8, Grate
		\\ Kelvin Lin (linkk4)
		\\ Eric Chaput (chaputem)
		\\ Jin Liu (liu456)
}

\date{}

%% Comments

\usepackage{color}

\newif\ifcomments\commentstrue

\ifcomments
\newcommand{\authornote}[3]{\textcolor{#1}{[#3 ---#2]}}
\newcommand{\todo}[1]{\textcolor{red}{[TODO: #1]}}
\else
\newcommand{\authornote}[3]{}
\newcommand{\todo}[1]{}
\fi

\newcommand{\wss}[1]{\authornote{blue}{SS}{#1}}
\newcommand{\ds}[1]{\authornote{red}{DS}{#1}}
\newcommand{\mj}[1]{\authornote{red}{MSN}{#1}}
\newcommand{\cm}[1]{\authornote{red}{CM}{#1}}
\newcommand{\mh}[1]{\authornote{red}{MH}{#1}}

% team members should be added for each team, like the following
% all comments left by the TAs or the instructor should be addressed
% by a corresponding comment from the Team

\newcommand{\tm}[1]{\authornote{magenta}{Team}{#1}}


\begin{document}

\begin{table}[hp]
\caption{Revision History} \label{TblRevisionHistory}
\begin{tabularx}{\textwidth}{llX}
\toprule
\textbf{Date} & \textbf{Developer(s)} & \textbf{Change}\\
\midrule
September 26, 2016 & Eric Chaput & Created Draft 1\\
September 27, 2016 & Kelvin Lin & Modified Team Meeting Plan for Draft 2\\
September 28, 2016 & Kelvin Lin & Added Team Member Roles for Draft 2\\
September 28, 2016 & Kelvin Lin & Added Git Workflow Plan for Draft 2\\
September 29 2016 & Eric Chaput & Added to Sections 3,6, and 7\\
\bottomrule
\end{tabularx}
\end{table}

\newpage

\maketitle


%TODO: INSERT MORE HERE
Having stated the problem to be solved in the problem statement document, Team 8 will develop the Genetic Cars project to solve it. The following document details the development plan for Team 8's Genetic Cars project..

\section{Team Meeting Plan}

\subsection{Overview}
Team 8's meetings will resemble a modified version of daily scrum meetings; 
however, instead of holding daily scrum meetings, they will be held weekly. Team 
8 will structure 3 meetings each week, with the first meeting requiring full 
attendance of all members, and the remaining 2 requiring members to attend as 
needed. The first meeting will resemble the daily scrum meeting, while the 
latter two meetings will resolve issues and difficulties mentioned in the former 
meeting.

\subsection{Meeting Structure}
Team meetings will be held three times per week: on Monday from 7:00PM to 
8:00PM, on Wednesday from 8:30AM to 10:30AM, and on Fridays from 2:30PM to 
4:30PM. On Mondays, the team meetings will be held in a meeting room at McMaster 
University. Kelvin Lin will be responsible for booking the meeting rooms, and 
releasing the meeting location 1 week before the meeting date. The meetings on 
Wednesday and Friday will be held in ITB 236.

The meetings on Mondays will be used to aggregate the achievements and 
difficulties of the previous week, and to plan for the upcoming week. Members 
will be presented with the opportunity to discuss any achievements and 
difficulties they encountered while working on their tasks. A list of 
difficulties 
will be created; however, there will not be an in-depth investigation or 
resolution of the aforementioned difficulty. For each issue mentioned, a list of 
team members who are qualified to solve the problem will be created, and a 
member from that list will be chosen to investigate that problem. A meeting will 
be held on the subsequent Wednesday or Friday in order to discuss the resolution 
or further issues pertaining to that problem. The meetings on Monday will 
conclude with a list of tasks each member is responsible for in the upcoming 
week.

The meetings on Wednesdays and Fridays will be used to address specific 
decisions pertaining to the project deliverables. Topics will include making 
specific design decisions, resolving issues mentioned on Monday's meeting, 
resolving any issues discovered throughout the week, and addressing any 
intergroup problems. To account for in-lab activities, the meetings will be 
divided into sub-meetings, and members will only be required to attend meetings 
that concern aspects of the tasks that they are working on. Members can work on 
the in-lab activities when they are not in a meeting. 

\subsection{Components of Meetings}
Key components of team meetings include the chair, the agenda, the scribe, and 
the meeting minutes.

\subsubsection{The Chair}
The chair will be responsible for organizing the topics to be discussed at the 
meeting and orchestrating the flow of the meeting. Before each meeting, the 
chair is responsible for creating an agenda for the meeting. During the 
meetings, the chair has an obligation to maintain a neutral front; the chair 
exert his authority to favour one side of an issue over the other side of the 
issue. However, the chair will have the authority to defer topics that threaten 
the overall structure of the meeting: if a topic exceeds the allotted time for 
discussion or if an unforeseen issue arises from a predetermined topic, the 
chair has the authority to defer the topic to a future meeting. The chair will 
not have the authority to veto decisions made by the team, nor will the chair's 
opinion be valued more than the other members. At the end of each meeting, the 
chair is responsible for establishing the next meeting date and making the 
appropriate reservations for a meeting space. Unless otherwise noted, Kelvin Lin 
will be the chair of the meetings.

\subsubsection{The Agenda}
The agenda is a written document to be written in \LaTeX. It details the topics 
to be covered in an upcoming meeting. At a minimum, it must provide a numbered 
list of topics to be covered, as well as a brief summary of each topic. The 
summary can be a point list. The agenda should be released to all team members 
before each meeting.

\subsubsection{The Scribe}
The scribe of the meeting will be responsible for taking appropriate notes at 
the meeting. Before each meeting, the scribe must prepare the appropriate 
equipment needed to take notes. This could include, but is not limited to, 
laptops, notebooks, pens, pencils, and paper. If possible, the scribe should 
obtain a copy of the agenda from the chair. During each meeting, the scribe has 
the obligation to keep accurate, unbiased notes that provide a faithful 
representation of the meeting. All events and decisions must be recorded in a 
manner so that they can be referenced in the future. Attendance, and future 
meeting dates should also be recorded. After each meeting, the scribe is 
responsible for formalizing the notes into meeting minutes. Unless otherwise 
noted, Jin Liu will be the scribe of the meetings.

\subsubsection{The Meeting Minutes}
The meeting minutes is a written document to be written in \LaTeX. It is a 
formalization of the notes taken by the scribe during each meeting. The meeting 
minutes can be appended to the corresponding meeting agenda of the same date. At 
a minimum, the meeting minutes must contain three sections: a section to detail 
the key points of the conversation during each topic of discussion, a summary of 
the key decisions made during each meeting, and a list of tasks each member is 
responsible for before the next meeting. The meeting minutes should be released 
to all team members before the start of the next meeting.

\subsection{Emergency Provisions}
In the unlikely event of an emergency, impromptu meetings can be held on 
Thursdays 12:30PM to 1:30PM, at a location to be determined. All members are 
expected to be present at emergency meetings. Depending on the urgency of the 
meeting, an agenda for the meeting may not exist; however, the scribe will still 
be responsible for taking notes and releasing meeting minutes after the meeting. 
Topics at emergency meetings will only pertain to issues at-hand: that is, no 
long-term planning will be done at emergency meetings unless otherwise required 
by the deliverable. If required, additional meetings can be scheduled through 
the Facebook Messenger platform on an as-needed basis.

\section{Team Communication Plan}
\subsection{Online Platforms}
Team 8 will communicate using a variety of online services and in-person 
meetings. While the structure of in-person meetings is described above, 


Team communication is to be facilitated online and in person. Team meetings will 
be conducted as outlined in section 1 of this development plan. Documents are to 
be shared through GitLab as is dictated by the course. Facebook group chat will 
be used as the primary means of communication, and all group members are aware 
of fellow members’ e-mails and phone numbers if those are necessary.

\subsubsection{Facebook}
Facebook serves as an ideal means of instant communication with group members when non technical communication is required. It also allows for a more informal communication enviroment then the other communication platforms. Group chat is to be used to communicate sudden updates to the group, and person to person messaging will be used to coordinate with individual group members.

\subsubsection{GitLab}
GitLab is to be used for formal communications that need to be archived for the purposes of this project. This includes the storing of meeting minutes but also the storing of comments in documents. GitLab is Team 8's primary method of document sharing between group memebers and its primary method of communicaition with McMaster staff.

\subsubsection{Phone}
While phone numbers have been exchanged between group members, it has been made clear that this method of communication is to be used only in the event of an emergency (i.e. a home power outage that disables online methods of communications), or in the event of a very sudden need (i.e. a last minute change in meeting location).

\section{Team Member Roles}
Team 8 will have the following roles: team lead, technology expert, graphics 
expert, theory expert, qualitative testing expert, and quantitative testing 
expert. The assignment and descriptions of each role follow.

\subsection{Assignment of Roles}
The following table shows the assignment of roles in Team 8.

%Table generated with: http://www.tablesgenerator.com/#
\begin{table}[h!]
	\begin{tabular}{ll}
		Team Member & Role                                        \\
		Kelvin Lin  & Team Lead, Technology Expert                \\
		Eric Chaput & Theory Expert, Quantitative Testing Expert  \\
		Jin Liu     & Graphics Expert, Qualitative Testing Expert
	\end{tabular}

	\caption{Team 8 Role Assignments}
\end{table}

\subsection{Description of Roles}

\subsubsection{Team Lead}
The team lead is responsible for the overall management of the project. The team 
lead will be responsible for ensuring that all deliverables are made to 
specification and on time. To accomplish this, the team lead must be sensitive 
to the member's skills and abilities, and partition tasks to allow all of the 
other members to succeed and to develop new skills. The team lead shall ensure 
that every member is aware of the tasks that need to be done, and what their 
role is on the team. The team lead is also responsible for facilitating 
communication to external parties by representing the voice of the team.

\subsubsection{Technology Expert}
The technology expert is responsible for maintaining current knowledge on 
general aspects of the technology used. The technology expert will serve as the 
first point of contact for any technical problems encountered with, but not 
limited to, Git, HTML5, JavaScript, \LaTeX, GanttProject, Doxygen, and Karma. 
The technology expert will be in charge of general troubleshooting, and he will 
redirect members to the appropriate experts when required. Moreover, the 
technology expert will be responsible for researching new technologies that can 
be used for the project.

\subsubsection{Graphics Expert}
The graphics expert is responsible for maintaining current knowledge on the 
algorithms and technology needed to implement the graphics for the project. The 
graphics expert shall learn all tools necessary in order to create a graphical 
interface for the project, and he will be responsible for communicating this 
knowledge to other members of the team in a manner comprehensible by both 
members with and without prior technical knowledge. The graphics expert will 
also be responsible for experimenting with new graphical technologies, and 
spearheading the graphics-related implementation initiatives of the project. 

\subsubsection{Theory Expert}
The theory expert will be responsible for maintaining current knowledge of the 
theory needed to comprehend the algorithms used in the project. The theory 
expert will be responsible for understanding theory behind the physics, learning 
algorithm, and graphics. The theory expert will be responsible for finding any 
references to published text when needed. Moreover, the theory expert will be 
responsible for communicating this knowledge to other members of the team in a 
manner comprehensible by both members with and without prior technical 
knowledge.

\subsubsection{Qualitative Testing Expert}
The qualitative testing expert will be responsible for maintaining current 
knowledge of qualitative testing methodologies within the scope of the project. 
The qualitative testing expert will be responsible for leading any qualitative 
tests including but not limited to market research and observation, in-person 
and online surveys, discussion/focus groups, and code review sessions. Such 
tasks may 
involve partitioning tasks for each initiative and assigning the tasks to the 
appropriate member. Moreover, the qualitative testing expert will be responsible 
for reporting any results to the group and making recommendations based on the 
results to improve the projects.

\subsubsection{Quantitative Testing Expert}
The quantitative testing expert will be responsible for maintaining current 
knowledge of quantitative testing methodologies within the scope of the project. 
Before the implementation of the project, the quantitative testing expert shall 
familiarize himself with different tools used to test the agreed-upon 
technologies. The quantitative testing expert will be responsible for leading 
any quantitative testing initiatives including unit testing, integration 
testing, system testing, automated testing, and manual testing. Such tasks may 
involve partitioning tasks for each initiative and assigning the tasks to the 
appropriate member. Moreover, the quantitative testing expert will be 
responsible for reporting any results to the group and making recommendations 
based on the results to improve the project.

\section{Git Workflow Plan}
Team 8 will utilize a feature-branch workflow plan. 

\subsection{Feature-branch Overview}
The feature-branch workflow branch is a Git workflow plan that utilizes 
different branches to implement different feature of the programs. Once a 
particular feature of the program is completed, its branch will be merged with 
the master branch. Labels will be used to tag submission files, while milestones 
will be used in order to signal any major revisions on the master branch.

\subsection{Rationale}
In selecting a Git workflow plan, Team 8 wanted a workflow that would provide 
maximum support for separation of concerns, while guaranteeing that there will 
always be a working version of the project easily assessable. Such workflow 
would simplify both the development, testing, and deployment efforts of the 
team. A high level of separations of concerns would allow multiple members to 
code concurrently without interference from code changes from other members. 
Members can also test their code using the master branch as a driver. This is a 
valid approach as any functional code on the master branch would have already 
been tested. The second requirement of the Team's desired workflow plan 
guarantees this by specifying that only functional versions of the project are 
to be committed to the master branch. 

The feature-branch workflow plan fully satisfies the requirements mentioned 
above. It supports separation of concerns by placing each feature of the program 
on a separate branch. Each branch will stem from a node on the master branch, 
allowing each branch to have an independent copy of the previously functional 
code. This will allow members to write new features in isolation, without having 
to account for continuous changes made by other members. The master branch can 
also serve as a driver for unit testing, and for regression testing. 
Accordingly, the feature-branch workflow was chosen.

\section{Proof of Concept Demonstration Plan}

\section{Technology}

This project will use technology that all group members are already familiar 
with on a basic level. The project itself will be coded in Java Script, however 
all group members understand that Java may be used if Java Script proves too 
difficult to use for this project. All group members are familiar with both of 
these languages and no further learning will be required. Grate’s graphics 
expert is also adept at creating graphics in Java Script. Written reports and 
documents will be generated through Latex, a technology being taught in the 3xa3 
course, and documents and reports will be shared through Git, another technology 
being taught in the 3xa3 course. The technology expert will ensure that all 
group members are aware of how to use these technologies. Testing will be aided 
by the use of Java’s built in unit test functionality.

\subsection{Java Script}

Java Script as a programming language is most often used for web applications, but not always. Java Script is the ideal programming language for this project for several reasons. Member's of team 8 are very familiar with Java Script as a language, and it is one of the few programming languages that team 8's graphics expert feels will work for this project. Our final project will also closely resemble a standard web page in design, (indeed the original source of our project is evidently suited for Java Script. Finally, the original basis for the project is also written in Java Script and this code can be more easily used for inspiration and trouble shooting for this reason.

\subsubsection{Possible switch to Java}

While team 8 is very familiar with Java Script, it is possible that Java Script may reveal itself to be unsuitable for this project. In the event that this occurs, team 8 is well prepared to transition to the Java programming language, as it is one with which we are more familiar. java was chosen as a back-up language as it is the programming language that team 8 is the most familiar with and is very easy to convert to from Java Script so that we do not lose progress already made. The need for a switch from the Java Script programming language is believed to be highly unlikely however, and this is merely a back-up contingency.

\subsection{Latex}

Latex will be used to generate documents for this project. Latex's ability for multiple team members to effect changes simultaneously and its versatility across platforms makes it ideal for this project. As of September 28 2016 all team members have aquired a working knowledge of how Latex functions, and it is believed that no further education will be required. If any such education is necessary TAs and the Latex tutorial found online will be team 8's go-to instructors. 

\subsection{GitLab}

GitLab will be used to share and store documents for this project. More information on GitLab and how it will be used for this project can be found in sections 2.1.2 and in the Git Workflow Plan found in section 4. All team 8 members are familiar with GitLab and how it functions. 

\section{Coding Style}

Coding will be styled primarily as has been taught by the software engineering 
program at McMaster University. Grate will undertake an object oriented 
programming approach to achieve maximum modularity and information hiding. The 
team shall also ensure that the code meets several aesthetic guidelines outlined 
here. All code is to be well commented, with comments being placed before the 
code. Comments are to be written with the understanding that they should be 
concise and easy to comprehend by outside parties. 

\subsection{Mozilla developer network}

Our formal coding style will be the Mozilla developer network, which contains subsections specifically for Java Script, our programming language of choice for this project. Specific guidelines can be found on the Mozilla developer network website.

\section{Project Schedule}

Provide a pointer to your Gantt Chart.

\section{Project Review}

\end{document}
