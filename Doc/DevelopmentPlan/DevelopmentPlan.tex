\documentclass{article}

\usepackage{booktabs}
\usepackage{tabularx}

\title{SE 3XA3: Development Plan\\Genetic Cars}

\author{Team 8, Grate
		\\ Kelvin Lin (linkk4)
		\\ Eric Chaput (chaputem)
		\\ Jin Liu (liu456)
}

\date{}

%% Comments

\usepackage{color}

\newif\ifcomments\commentstrue

\ifcomments
\newcommand{\authornote}[3]{\textcolor{#1}{[#3 ---#2]}}
\newcommand{\todo}[1]{\textcolor{red}{[TODO: #1]}}
\else
\newcommand{\authornote}[3]{}
\newcommand{\todo}[1]{}
\fi

\newcommand{\wss}[1]{\authornote{blue}{SS}{#1}}
\newcommand{\ds}[1]{\authornote{red}{DS}{#1}}
\newcommand{\mj}[1]{\authornote{red}{MSN}{#1}}
\newcommand{\cm}[1]{\authornote{red}{CM}{#1}}
\newcommand{\mh}[1]{\authornote{red}{MH}{#1}}

% team members should be added for each team, like the following
% all comments left by the TAs or the instructor should be addressed
% by a corresponding comment from the Team

\newcommand{\tm}[1]{\authornote{magenta}{Team}{#1}}


\begin{document}

\begin{table}[hp]
\caption{Revision History} \label{TblRevisionHistory}
\begin{tabularx}{\textwidth}{llX}
\toprule
\textbf{Date} & \textbf{Developer(s)} & \textbf{Change}\\
\midrule
September 26, 2016 & Eric Chaput & Created Draft 1\\
September 27, 2016 & Kelvin Lin & Modified Team Meeting Plan for Draft 2\\
\bottomrule
\end{tabularx}
\end{table}

\newpage

\maketitle


%TODO: INSERT MORE HERE
The following document details the development plan for Team 8's Genetic Cars.

\section{Team Meeting Plan}

\subsection{Overview}
Team 8's meetings will resemble a modified version of daily scrum meetings; 
however, instead of holding daily scrum meetings, they will be held weekly. Team 
8 will structure 3 meetings each week, with the first meeting requiring full 
attendance of all members, and the remaining 2 requiring members to attend as 
needed. The first meeting will resemble the daily scrum meeting, while the 
latter two meetings will resolve issues and difficulties mentioned in the former 
meeting.

\subsection{Meeting Structure}
Team meetings will be held three times per week: on Monday from 7:00PM to 
8:00PM, on Wednesday from 8:30AM to 10:30AM, and on Fridays from 2:30PM to 
4:30PM. On Mondays, the team meetings will be held in a meeting room at McMaster 
University. Kelvin Lin will be responsible for booking the meeting rooms, and 
releasing the meeting location 1 week before the meeting date. The meetings on 
Wednesday and Friday will be held in ITB 236.

The meetings on Mondays will be used to aggregate the achievements and 
difficulties of the previous week, and to plan for the upcoming week. Members 
will be presented with the opportunity to discuss any achievements and 
difficulties he encountered while working on his tasks. A list of difficulties 
will be created; however, there will not be an in-depth investigation or 
resolution of the aforementioned difficulty. For each issue mentioned, a list of 
team members who are qualified to solve the problem will be created, and a 
member from that list will be chosen to investigate that problem. A meeting will 
be held on the subsequent Wednesday or Friday in order to discuss the resolution 
or further issues pertaining to that problem. The meetings on Monday will 
conclude with a list of tasks each member is responsible for in the upcoming 
week.

The meetings on Wednesdays and Fridays will be used to address specific 
decisions pertaining to the project deliverables. Topics will include making 
specific design decisions, resolving issues mentioned on Monday's meeting, 
resolving any issues discovered throughout the week, and addressing any 
intergroup problems. To account for in-lab activities, the meetings will be 
divided into sub-meetings, and members will only be required to attend meetings 
that concern aspects of the tasks that they are working on. Members can work on 
the in-lab activities when they are not in a meeting. 

\subsection{Components of Meetings}
Key components of team meetings include the chair, the agenda, the scribe, and 
the meeting minutes.

\subsubsection{The Chair}
The chair will be responsible for organizing the topics to be discussed at the 
meeting and orchestrating the flow of the meeting. Before each meeting, the 
chair is responsible for creating an agenda for the meeting. During the 
meetings, the chair has an obligation to maintain a neutral front; the chair 
exert his authority to favour one side of an issue over the other side of the 
issue. However, the chair will have the authority to defer topics that threaten 
the overall structure of the meeting: if a topic exceeds the allotted time for 
discussion or if an unforeseen issue arises from a predetermined topic, the 
chair has the authority to defer the topic to a future meeting. The chair will 
not have the authority to veto decisions made by the team, nor will the chair's 
opinion be valued more than the other members. At the end of each meeting, the 
chair is responsible for establishing the next meeting date and making the 
appropriate reservations for a meeting space. Unless otherwise noted, Kelvin Lin 
will be the chair of the meetings.

\subsubsection{The Agenda}
The agenda is a written document to be written in \LaTeX. It details the topics 
to be covered in an upcoming meeting. At a minimum, it must provide a numbered 
list of topics to be covered, as well as a brief summary of each topic. The 
summary can be a point list. The agenda should be released to all team members 
before each meeting.

\subsubsection{The Scribe}
The scribe of the meeting will be responsible for taking appropriate notes at 
the meeting. Before each meeting, the scribe must prepare the appropriate 
equipment needed to take notes. This could include, but is not limited to, 
laptops, notebooks, pens, pencils, and paper. If possible, the scribe should 
obtain a copy of the agenda from the chair. During each meeting, the scribe has 
the obligation to keep accurate, unbiased notes that provide a faithful 
representation of the meeting. All events and decisions must be recorded in a 
manner so that they can be referenced in the future. Attendance, and future 
meeting dates should also be recorded. After each meeting, the scribe is 
responsible for formalizing the notes into meeting minutes. Unless otherwise 
noted, Jin Liu will be the scribe of the meetings.

\subsubsection{The Meeting Minutes}
The meeting minutes is a written document to be written in \LaTeX. It is a 
formalization of the notes taken by the scribe during each meeting. The meeting 
minutes can be appended to the corresponding meeting agenda of the same date. At 
a minimum, the meeting minutes must contain three sections: a section to detail 
the key points of the conversation during each topic of discussion, a summary of 
the key decisions made during each meeting, and a list of tasks each member is 
responsible for before the next meeting. The meeting minutes should be released 
to all team members before the start of the next meeting.

\subsection{Emergency Provisions}
In the unlikely event of an emergency, impromptu meetings can be held on 
Thursdays 12:30PM to 1:30PM, at a location to be determined. All members are 
expected to be present at emergency meetings. Depending on the urgency of the 
meeting, an agenda for the meeting may not exist; however, the scribe will still 
be responsible for taking notes and releasing meeting minutes after the meeting. 
Topics at emergency meetings will only pertain to issues at-hand: that is, no 
long-term planning will be done at emergency meetings unless otherwise required 
by the deliverable. If required, additional meetings can be scheduled through 
the Facebook Messenger platform on an as-needed basis.

%Kelvin Lin, as team leader will also serve as 
%the chair for meetings. Eric Chaput will take notes and record minutes during 
%these meetings as Grate’s scribe. Agendas for each meeting are to be determined 
%at the end of the previous meeting (i.e. Friday’s agenda is to be determined at 
%the end of Wednesday’s meeting), and rooms for the Monday meeting are to be 
%booked during Friday’s meeting. All group members have agreed to hold impromptu 
%meetings on Saturday’s should a group emergency call for them (although this is 
%believed to be highly unlikely).

\section{Team Communication Plan}
\subsection{Online Platforms}
Team 8 will communicate using a variety of online services and in-person meetings. While the structure of in-person meetings are described above, 


Team communication is to be facilitated online and in person. Team meetings will 
be conducted as outlined in section 1 of this development plan. Documents are to 
be shared through GitLab as is dictated by the course. Facebook group chat will 
be used as the primary means of communication, and all group members are aware 
of fellow members’ e-mails and phone numbers if those are necessary.

\section{Team Member Roles}

\section{Git Workflow Plan}

\section{Proof of Concept Demonstration Plan}

\section{Technology}

This project will use technology that all group members are already familiar 
with on a basic level. The project itself will be coded in Java Script, however 
all group members understand that Java may be used if Java Script proves too 
difficult to use for this project. All group members are familiar with both of 
these languages and no further learning will be required. Grate’s graphics 
expert is also adept at creating graphics in Java Script. Written reports and 
documents will be generated through Latex, a technology being taught in the 3xa3 
course, and documents and reports will be shared through Git, another technology 
being taught in the 3xa3 course. The technology expert will ensure that all 
group members are aware of how to use these technologies. Testing will be aided 
by the use of Java’s built in unit test functionality.

\section{Coding Style}

Coding will be styled primarily as has been taught by the software engineering 
program at McMaster University. Grate will undertake an object oriented 
programming approach to achieve maximum modularity and information hiding. The 
team shall also ensure that the code meets several aesthetic guidelines outlined 
here. All code is to be well commented, with comments being placed before the 
code. Comments are to be written with the understanding that they should be 
concise and easy to comprehend by outside parties. Brackets will not receive 
their own line simply for the sake of consistency across the entire group’s 
work.

\section{Project Schedule}

Provide a pointer to your Gantt Chart.

\section{Project Review}

\end{document}

